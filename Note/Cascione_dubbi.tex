\documentclass[Cascione]{subfiles}
\begin{document}
\section{Chiarimento sugli operatori di Green per le Ode}
Dalle fonti ( principalmente \url{http://www.nada.kth.se/~annak/greens1d_odes.pdf}) ho capito che le funzioni di Green per una ODE di secondo grando le costruisco a partire da due soluzioni $y_1, y_2$ linearmente independenti dell'equazione omogenea (operatore $L$).
Cioè:

\begin{displaymath}
	g^+ ( x \vert \xi) = \theta( x-\xi)  \left( c_1 y_1(x) + c_2 y_2(x)\right)
\end{displaymath}
\begin{displaymath}
	g^- ( x \vert \xi) = \theta( \xi-x)  \left( d_1 y_1(x) +d_2  y_2(x)\right)	
\end{displaymath}
Scegliendo $a_1,b_1,a_2,b_2$ in modo che le due funzioni si raccordino in $\xi$ ma le derivate prime presentino un salto:
			\begin{equation}
			\begin{cases}
						y_1(\xi) (c_1-d_1) + y_2(\xi) ( c_2 - d_2) = 0 \\
						\dot{y}_1(\xi) (c_1-d_1) + \dot{y}_2(\xi) ( c_2 - d_2) = -1 \\
            \end{cases}
			\end{equation}

Tali funzioni di green sono uniche solo se l'opertore $L$ è anche corredato di condizioni al bordo
\vspace{1mm}
Quindi le funzioni di green di un opertare differenziale ordinario di secondo ordine:
\begin{displaymath}
L = \partial_x^2 + \alpha \partial_x + \beta
\end{displaymath}
sono tante quanti le possibili coppie di dati sul bordo.
\vspace{2mm}
Dubbio: allora come è fatto l'unico operatore di Green $\pm$ associato all'operatore $L$?\\
Ho detto che 
\begin{displaymath}
	G_\pm \psi(x) = \int_\Real g^\pm(x \vert \xi) \psi(\xi) d\xi
\end{displaymath}
ma senza aver specificato dei dati al bordo $g^\pm$ non è univocamente definita.

\end{document}