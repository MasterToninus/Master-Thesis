\documentclass{article}
\usepackage{titling}
\usepackage{blindtext}
%---------------------------------------------------------------------------------------------------------------------------------------------------------------------
\title{Demystifing the Peierls' Brackets Construction}
\author{Antonio Michele Miti}
\date{\vspace{-5ex}} % workaround to omit the Date !

\begin{document}
\begin{titlingpage}
    \maketitle
    \begin{abstract}

The aim of this talk is to present an extension and modernization of
         \emph{Peierls' algorithm}, a procedure originally proposed by Rudolf E. Peierls in a seminal paper dated 1952. 
    This is an effective, but rather convoluted, "recipe"  to prescribe a pre-symplectic structure on the space of dynamical configurations for a classical fields theory.
    \\
        Considering a suitable class of  abstract dynamical systems, it is possible to reformulate more rigorously the original algorithm in every step allowing a further extension to non-linear systems.
        This class is not trivial and encompass most of the interesting classical mechanical system
        regardless of which is the cardinality of the degrees of freedom.
        \\
        In general, this construction determines a binary operation called the \emph{Peierls brackets}.
        On condition of considering a suitable quotientation of its domain, this object yields the  field-theoretic pre-symplectic structure often postulated in the modern bibliography, 
        \\
        The result is of great importance from what concerns the algebraic quantum field theory since 
        the identification of this "pre-quantum" symplectic space takes a pivotal role in the realisation of the related scheme of quantization.
       This construction is also particularly interesting from the point of view of geometric mechanics because represents a unified language for treating discrete and continuous degrees mechanicals system on a same common ground.

    \end{abstract}
\end{titlingpage}
\end{document}