\documentclass{beamer}
	\title{La mia prima presentazione}
	\author{Antonio Michele Miti}
	\date{20 novembre 2015}


\begin{document}

%\/\/\/\/\/\/\/\/\/\/\/\/\/\/\/\/\/\/\/\/\/\/\/\/\/\/\/\/\/\/\/\/\/\/\/\/\/\/\/\/\/\/\/\/\/\/\/\/\/\/\/\/\/\/\/\/\/\/\/\/\/\/\/\/\/\/\/\/\/
%				Intro
%\/\/\/\/\/\/\/\/\/\/\/\/\/\/\/\/\/\/\/\/\/\/\/\/\/\/\/\/\/\/\/\/\/\/\/\/\/\/\/\/\/\/\/\/\/\/\/\/\/\/\/\/\/\/\/\/\/\/\/\/\/\/\/\/\/\/\/\/\/
	\begin{frame}
		\maketitle
	\end{frame}
	
	\begin{frame}
		\frametitle{Intro}
		(struttura della tesi vs struttura della presentazione)
	\end{frame}
	

%\/\/\/\/\/\/\/\/\/\/\/\/\/\/\/\/\/\/\/\/\/\/\/\/\/\/\/\/\/\/\/\/\/\/\/\/\/\/\/\/\/\/\/\/\/\/\/\/\/\/\/\/\/\/\/\/\/\/\/\/\/\/\/\/\/\/\/\/\/
%				Parte 1 : 	"Foundations of Mechanics"
%\/\/\/\/\/\/\/\/\/\/\/\/\/\/\/\/\/\/\/\/\/\/\/\/\/\/\/\/\/\/\/\/\/\/\/\/\/\/\/\/\/\/\/\/\/\/\/\/\/\/\/\/\/\/\/\/\/\/\/\/\/\/\/\/\/\/\/\/\/		
	\part{Meccanica Geometrica}
		\frame{\partpage}
	
		\begin{frame}
		\frametitle{IDEA Meccanica Geometrica:}
			\begin{itemize}
				\item<1-> Raccontare l'approccio canonico coordinate free alla meccanica classica a finiti gradi
					\begin{itemize}
						\item Lo spazio delle fasi
						\item La forma simplettica
						\item lagrangiana hamiltoniana legendre
						\item struttura di Poisson
					\end{itemize}
				\item<1-> Dire che un suo aspetto importante è che funge da base per le teorie di quantizzazione
				\item<1-> Ci interessa quantizzare i campi quindi ci interessa il formalismo canonico per i campi
				\item<2-> Presentare l'approccio al formalismo canonico dei campi in 2 step:
					\begin{itemize}
						\item Passaggio Canonico Ordinario a Canonico Covariante (a finiti gradi)
						\item Passaggio Canonico Covariante a Finiti Gradi a Canonico Covarianti a gradi continuo
						\item (extra) Recupare il formalismo canonico non covariante anche per i campi costruendo il dato su una superficie di Cauchy (ricordare che al continuo ci sono controesempi che darboux non vale)
					\end{itemize}
				\item<3-> Ok, fin qui abbiamo un insieme. Come lo doto di una struttura simplettica?
			\end{itemize}
	\end{frame}
	
	\begin{frame}
	\frametitle{Formalismo canonico}
		(usually approach) I passi principali per la costruzione del formalismo canonico per teorie con $N$ gradi di libertà

		\end{frame}
		
		\begin{frame}
			\frametitle{Formalismo Canonico Covariante}
		
		\end{frame}
	



%\/\/\/\/\/\/\/\/\/\/\/\/\/\/\/\/\/\/\/\/\/\/\/\/\/\/\/\/\/\/\/\/\/\/\/\/\/\/\/\/\/\/\/\/\/\/\/\/\/\/\/\/\/\/\/\/\/\/\/\/\/\/\/\/\/\/\/\/\/
%				Parte 2 : 	"L'algoritmo di Peierls"
%\/\/\/\/\/\/\/\/\/\/\/\/\/\/\/\/\/\/\/\/\/\/\/\/\/\/\/\/\/\/\/\/\/\/\/\/\/\/\/\/\/\/\/\/\/\/\/\/\/\/\/\/\/\/\/\/\/\/\/\/\/\/\/\/\/\/\/\/\/		
	\part{Costruzione della struttura simplettica nel caso dei campi classici}
	\frame{\partpage}
	
	\begin{frame}
		\frametitle{ IDEA La forma simpettica per i sistemi campo Classico }
			\begin{itemize}
				\item semplificare: usare i grafici già fatti e considerare solo i campi scalari
			\end{itemize}
	\end{frame}
	
	
%\/\/\/\/\/\/\/\/\/\/\/\/\/\/\/\/\/\/\/\/\/\/\/\/\/\/\/\/\/\/\/\/\/\/\/\/\/\/\/\/\/\/\/\/\/\/\/\/\/\/\/\/\/\/\/\/\/\/\/\/\/\/\/\/\/\/\/\/\/
%				Parte 3 : 	"Realizzazione Specifica del campo di Jacobi"
%\/\/\/\/\/\/\/\/\/\/\/\/\/\/\/\/\/\/\/\/\/\/\/\/\/\/\/\/\/\/\/\/\/\/\/\/\/\/\/\/\/\/\/\/\/\/\/\/\/\/\/\/\/\/\/\/\/\/\/\/\/\/\/\/\/\/\/\/\/		
	\part{Realizzazione per il Caso specifico dei campi di Jacobi}
	\frame{\partpage}

	\begin{frame}
		\frametitle{ IDEA }
			\begin{itemize}
				\item ricordare le basi, varietà riemmaniana, metrica ,geodetiche jacobi
			\end{itemize}
	\end{frame}


%\/\/\/\/\/\/\/\/\/\/\/\/\/\/\/\/\/\/\/\/\/\/\/\/\/\/\/\/\/\/\/\/\/\/\/\/\/\/\/\/\/\/\/\/\/\/\/\/\/\/\/\/\/\/\/\/\/\/\/\/\/\/\/\/\/\/\/\/\/
%				Fine	!!!
%\/\/\/\/\/\/\/\/\/\/\/\/\/\/\/\/\/\/\/\/\/\/\/\/\/\/\/\/\/\/\/\/\/\/\/\/\/\/\/\/\/\/\/\/\/\/\/\/\/\/\/\/\/\/\/\/\/\/\/\/\/\/\/\/\/\/\/\/\/		
	\begin{frame}
		\frametitle{ Conclusioni }
			\begin{itemize}
				\item Interpretazione geometrica: punto chiave: ci sono 2 spazi la cui geometria soggiace al discorso: spazio delle soluzioni e spazio delle lagrangiane
			\end{itemize}
	\end{frame}
	
\end{document}


		\begin{enumerate}
			\item si introducono la coordinate di configuazione $q^j$ e i momenti canonici coniugati $p^j$
			\item si definisce la 2-forma $\omega = dp^i \wedge dq^i$
			\item combinando $p^i,q^j$ in un'unica variablie $Q^I , I = 1,\ldots,2N$, con $Q^i=p^i$ per $i\leq\leq N$ e $Q^i= q^{i-N}$ per $i>N$. E' possibile riguardare $\omega$ come una matrice $2N\times \N$ antisimmetrica i cui elementi non nulli sono
				\begin{displaymath}
					\omega_{i , i+N} = -\omega_{i+N,i} = 1
				\end{displaymath}
			\item le parentesi di poisson per 2 funzioni $A(Q^I) , B(Q^J)$ vengono definite come:
				\begin{displaymath}
						\left[ A , B \right] = \omega^{I J} \frac{partial A}{\partial Q^I} \frac{\partial B}{\partial Q^J}					
				\end{displaymath}
		\end{enumerate}