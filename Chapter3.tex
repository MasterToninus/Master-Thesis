\documentclass[Main]{subfiles}
\begin{document}

\chapter{Algebraic Quantization}
The point we want to get, that we will face in the next chapter, is the algebraic quantization of geodesic system.
For this purpose it is necessary to devote a chapter to the description of algebraic quantization scheme.
We will show two realizations of the scheme applicable to a class of systems sufficiently broad to encompass the system that we want to examine.

\section{Overview on the Algebraic Quantization Scheme.}
Contemporary quantum field theory is mainly developed as quantization of classical fields. Classical field theory thus is a necessary step towards quantum field theory.\danger \footnote{Cito testualmente Mangiarotti, shardanashivly}
The \emph{"Quantization process"} has to be considered as an algorithm , in the sense of self-conteining succession of instruction, that has to be performed in order to establish a correspondence between a classical field theory and its quantum counterpart.
\danger\footnote{forse l'nlab esprime la cosa meglio di me \url{http://ncatlab.org/nlab/show/quantization}. Sono d'accordo con il loro approccio ma non voglio usare la loro formulazione perchè in fondo ci sono arrivato anche da solo :P}

On this basis the axiomatic theory of quantum fields takes the role of "validity check". It provide a set of conditions that must be met in order to establish whether the result can be consider a proper quantum field theory.
Basically there are no physical/philosophical principles which justifies "a priori" the relation between mathematical objects (e.g the classical state versus quantum states) individually. The scheme can only be ratified " a posteriori" as whole verifying the agreement with the experimental observations.

However this is by no means different from what is discussed in ordinary quantum mechanics where there are essentially two plane:
the basic formalism of quantum mechanics, which is substantially axiomatic and permits to define an abstract quantum mechanical system, and the quantization process that determine how to construct the quantum analogous of a classical system realizing the basic axioms.

We refer to the algebraic quantization as a \emph{scheme of quantization} because it's not a single specific procedure but rather a class of algorithms.
These algorithms are the same concerning the quantization step per se (costruction of the *-algebra of classical observable) but they differ in the choice of the classical objects  (essentially the classical observables) to be subjected to the procedure.

Basically an algebraic quantization is achieved in three steps:
\begin{enumerate}
	\item \textbf{Classical Step}\\
		Identify all the mathematical structures necessary to define the field, i.e. the couple $(E,P)$.\\
		In general every quantization process exploit some conditions on the quantum field structure that has to be met.
	\item \textbf{Pre-Quantum Step}\\
		\danger Are lowered from the top some additional mathematical over-structure  on the classic framework.
		
		
	\item \textbf{Quantization}\\
		bo
\end{enumerate}



As said previously, the realization of the Algebraic scheme are many: Fedosov's procedure, by Deformation, Peierls' procedure, by Initial Data etc .
In the next section we review the last two.

\section{Quantization with Peierls Bracket.}

	\subsection{Classical Step}
	Applicability of the procedure.
	
	\subsection{PreQuantum Step.}
	
	\subsection{Second Quantization Step.}


\section{Quantization by Initial Data.}

\end{document}
