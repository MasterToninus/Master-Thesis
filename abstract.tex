\documentclass[Main]{subfiles}
\begin{document} 
	\begin{abstract}
		The aim of this thesis is to extend and modernize the construction, originally proposed by Peierls in 1952, of the pre-symplectic structure for a classical fields theory.
		The identification of a symplectic space for fields is crucial ingredient in the scheme of algebraic quantization as it directly reflects on the definition of the canonical commutation rules on the *-algebra of quantum observables.
		Considering a suitable class of  abstract dynamical systems, it is possible to reformulate more rigorously the Peierls' algorithm in every step extending it also to non-linear systems.
		In the course of our work we have highlighted how this procedure intervenes in the realization of algebraic quantization, showing furthermore how this can be related to the \emph{initial data quantization}.
		At last we compute the Peierls bracket and the construct the pre-quantum symplectic space for the case of Jacobi fields, the linearization of the geodesic motion on a pseudo-Riemannian manifold, regarding it as a particular mechanical system.

	\end{abstract}

	\renewcommand{\abstractname}{Sommario}
	\begin{abstract}
	Lo scopo di questa tesi e di estendere e adattare al linguaggio moderno la costruzione, orginariamente proposta da Peierls nel 1952, della struttrura presimplettica associata ad una teoria di campo classica.
	L'identificazione di uno spazio simplettico per i campi è un ingrediente cruciale nello schema di quantizzazione algebrica in quanto si riflette direttamente sulla definizione delle regole di commutazione canonica sulla *-algebra degli osservabili quantistici.
	Considerando una opportuna classe di sistemi dinamici astratti, è possibile riformulare in modo più rigoroso l'algoritmo di Peierls in ogni suo passo estendendolo inoltre a sistemi non lineari.
	Nel corso del  nostro lavoro abbiamo sottilineato come questa preocedura interviene nella realizzazione della quantizzazione algebrica, mostrando inoltre come questa possa essere collegata alla \emph{quantizzazione per dati iniziali}.
	Alla fine calcoliamo le parentesi di Peierls e costruiamo lo spazio simplettico pre-quantistico per il caso dei campi di Jacobi, linearizzazione del moto geodetico su varietà pseudo-Riemanniane, riguardandandolo come un particolare sistema meccanico.
	
	
	
	\end{abstract}
\end{document}