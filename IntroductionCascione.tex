\documentclass[Main]{subfiles}
\begin{document}

\chapter*{Introduction}
\addcontentsline{toc}{chapter}{Introduction}
%
\ifToninus
\section{Motivazione della Tesi}
\begin{enumerate}
	\item La teoria dei campi è fantastica.. blabla  eccezionale verifica sperimentale .... modello standard
	\item ma fa a cazzotti con la relatività generale
	\item comunità è d'accordo con il fatto che serve una quantum gravity
	\item nel meantime ci si è dimenticati di dare un'occhiata al regime intermedio di teoria su spazio curvo. Anche se non è detto che questa possa avere ruolo nella grande unificazione è comunque un tema necessario per lo sviluppo della cosmologia e per il trattamento di corpi esotici come i buchi neri.
\end{enumerate}
\fi
		%
		%In che ambito lavori? quantizzazione algebrica su spazio tempo curvo
		%
Quantum electrodynamics and the so-called standard model of particles have been experimentally verified to an outstanding degree of precision and allowed us to have an almost fully satisfactory and unified description of the electro-weak forces. 
Although a quantum theory of gravity is needed in order to reconcile general relativity with the principles of quantum mechanics. 

Nonetheless, the quest to finding such \emph{theory of everything} often lead the community to neglect the existence of an intermediate regime, namely the quantum field theory in curved background, which is expected to provide an accurate description of quantum phenomena in a regions where the effects of curved spacetime may be significant, but effects of quantum gravity itself may be neglected. 
		%
		%Che problemi aperti ci sono? La caratterizzazione della struttura simplettica da attribuire ad una teoria classica da quantizzare, si usa perierls come ricetta ma non si sa quale origine abbia
		%
		%To this end, it is necessary to develop a formulation of quantum field theory, preferably based on a set of first principles, which can be applied to the largest possible class of backgrounds. 
		%
At the moment, Algebraic quantum field theory (AQFT) is proven to be the most promising way to reach a general and mathematical rigorous description of the foundations of quantum fields on a  sufficiently large class of (fixed) backgrounds.

The algebraic approach, as the majority of contemporary quantum field theory, is developed as quantization of classical fields.
		%
		%. In this case the quantization procedure is based on a set of first principles, essentially proposed by Dimock as an extension of axioms of Haag and Kastler stated on Minkowski spacetime.
		%
From that should be clear that classical field theory  is thus a necessary step towards the understanding of the foundations of  quantum field theory.
	%A classical field theory is essentially defined by a local variational principle for a given set of dynamical fields on a given spacetime manifold. The corresponding (Lagrangian) functional determines the equation of motion satisfied by the dynamical fields. 
	According to the canonical formalism of mechanics, any  classical physical system, fields included, could be encompassed by two object  called \emph{phase space} and \emph{classical observables algebra}. 
	From a mathematical point of view  the first is a symplectic manifold, namely a smooth manifold endowed with a non-degenerate 2-form, while the second is a Poisson algebra constituted by functionals on the Phase Space.
	The most common way to building these structures requires an explicit choice of a time function even when no one such choice is natural. 
	However, it is already know an alternative, completely covariant, construction based on the so-called \emph{covariant phase space} and \emph{Peierls brackets}.
	Each one leads to a different realization of the algebraic quantization scheme, the first is known ad \emph{quantization via initial data} while the second is called \emph{quantization via the Peierls brackets}.
	
The covariant phase space is defined as the \emph{space of dynamic configurations}, \textit{i.e.}  the infinite-dimensional space of solutions of equations of motion, while the \emph{Peierls brackets} provide an effective "recipe" to prescribe a pre-symplectic structure on this space, completing the simplectic picture for the system.
%Il problema delle peierls

Browsing through the literature, it is clear that the Peierls' construction never received particular attention since its debut in 1952 \cite{Peierls1952}.
This can be ascribed mainly to the lack of a convincing geometric interpretation
which had the effect of limiting its reception often relegating its role to that of a mere  "mathematical trick".

%Cosa vogliamo Fare
The aim of this thesis is to review the original Peierls' procedure in every single step adapting it to a more rigorous and modern mathematical formalism.
%Come lo Facciamo
To take a step closer to the comprehension of this object we study the well-known geodesic problem regarding it as a special case of a field-like system.
Essentially this example is noteworthy from two aspect:
\begin{enumerate}
	\item As a system with discrete degrees of freedom:
	\begin{itemize}
		\item The kinematic configurations are parametrized curves on a Riemannian manifold.
\ifToninus
				In other words this system is a very simple field with base manifold the real line $\Real$ and valued to the manifold $Q$. 
				In this sense it represent a complementary example to the basic real scalar field.
\fi

		\item The \emph{initial data} for systems of this type are simply pairs of finite-dimensional vectors.
\ifToninus
			According to the \emph{quantization procedure by initial data} the symplectic form associated to this system is unique and ,eventually, can be proved that it corresponds to that, generally different, constructed through the Peierls' method.
\fi

	\end{itemize}

	\item As a system dynamically ruled by the \emph{geodesic equation}:
	\begin{itemize}
		\item The equations of dynamics are generally higher non-linear. 
		Typically, any realizations of the algebraic quantization scheme, including the Peierls' algorithm, require to pass through the linearization of the equations of motion which takes the name of \emph{Jacobi equations}.
			\\
			The solutions of such linearization are called \emph{Jacobi fields},
			these are objects extensively studied from the point of view of differential geometry 
			( where they are introduced as a tangent field over a geodesic variation) but rarely are approached as a field-like dynamical system.
		\item The differential operator of Jacobi equations is \emph{normally hyperbolic}, this allows us to complete the quantization scheme according to two different procedure: 
		the first one uses the Peierls bracket, exploiting the Green hyperbolicity and formal self-adjointness properties, 
		the second is the initial data procedure, as hyperbolic operator in PDE sense.
	\end{itemize}
\end{enumerate}

Since the field of Jacobi lends itself to be quantized both according to the Peierls procedure than according the initial data procedure,
the comparison between the two symplectic forms thus obtained,
It allows us to assign a geometric interpretation to the original Peierls' method.

\vspace{3mm}
%Sinossi
We briefly summarize the content of the thesis.

The first chapter is devoted to reviewing the mathematical framework  underlying to the rigorous formulation of continuous classical system, starting point of every algebraic quantization realizations.
We begin by introducing the notion of \emph{smooth bundles}, these are the suitable objects to encode the kinematical structure of a generic field system.
Subsequently we define the notion of globally hyperbolic spacetime as  the natural arena for the mathematical theory of hyperbolic (systems of) partial differential equations, in which the Cauchy problem is well posed.
Finally we outline the theory of Green hyperbolic operators, the class of linear differential operators on a vector bundle to which the Peierls brackets construction as well as the corresponding quantization procedure applies.

Chapter 2 is dedicated to introducing the procedure of construction of the Peierls brackets.
In the first part we make use of the mathematical language developed above in order to formalize the correct abstract mechanical system for which it the Peierls procedure is well defined. Furthermore we will show how the most familiar mechanical systems, namely the point particle and the fundamental fields over a spacetime, can be treated in a unified way as special cases of the aforementioned abstract system.
\\
In the end we propose an extended version of the original Peierls' algorithm obtained combining the construction proposed in his paper\cite{Peierls1952} with some recent references, mainly \cite{Marolf1993}\cite{Dewitt1999}\cite{Forger2005}\cite{Sharan2010}\cite{Khavkine2014}.
Instead of limit ourselves to the case of scalar field only, we extend the step-by-step procedure proposed by Peierls to a large class of abstact mechanical systems, not necessarily linear.

In order to pursue the study of the system under investigation it is necessary to introduce the scheme of algebraic quantization.
To this end, the third chapter is focused to presenting two realizations of the algebraic quantization scheme.
The two are distinguished by the different construction of the pre-quantum symplectic space associated to the classical system.
\\
The first one is based on the restriction of the Peierls brackets to the class of \emph{classical observables}, the resulting symplectic form is sometimes prescribed axiomatically \cite{Dewitt1999}\cite{Esposito}\cite{Benini}.\\
The second exploits the hyperbolicity of equations of motion of the system and it is known as \emph{quantization} via the \emph{initial data}\cite{Wald1994}.

In the last chapter we will apply all the formalism developed so far to the case of the geodesic field.
We construct the Peierls bracket for such system and, as a  concrete example, we carry out all calculations for the specific case of a FRW spacetime spatially flat.
Then we will construct two equivalent pre-quantum symplectic space related to the geodesic system following the step-by-step algorithm presented in chapter 3 .
At last we propose a geometric picture of the whole Peierls construction.


%Commenti Miei
\ifToninus
\section{Intro Preliminare (OLD)}


%Inizio troppo deciso... manca completamente la contestualizzazione. 
%In che ambito lavori? quantizzazione algebrica su spazio tempo curvo
%Che problemi aperti ci sono? La caratterizzazione della struttura simplettica da attribuire ad una teoria classica da quantizzare, si usa perierls come ricetta ma non si sa quale origine abbia
% Che teoria usi in generale? 
%Quali sono i suoi ingredienti base? 
%La puoi delineare in poche parole? Le parentesi di Peierls non sono il punto di inizio, ma quello di mezzo.

La prima parte della tesi è stata rivolta allo studio del framework matematico necessario per dare una formulazione rigorosa dei sistemi classici continui, punto di partenza di ogni schema di quantizzazione algebrica.
Nello specifico viene fatta una digressione sui Fibrati Topologici e viene sfruttata la definizione di fibrato liscio per presentare l'approccio geometrico alla meccanica classica sia per sistemi a gradi di libertà finiti che continui.

Nella seconda parte viene presentato l'algoritmo di Peierls che rappresenta una “ricetta” efficace per attribuire una struttura pre-simplettica allo spazio delle configurazioni dinamiche di un sistema qualunque.
Dalla ricerca bibliografica è evidente come questo strumento a partire dal suo esordio (nel 1952) fino ad oggi non abbia mai ricevuto particolare attenzione. Questo sembra dovuto soprattutto alla mancanza di una convincente interpretazione geometrica.
\newline
Per fare un passo verso la comprensione di questo oggetto viene studiato l’estremamente noto problema della geodetica vedendolo come un sistema campo.
Emerge sin da subito come il calcolo delle parentesi Peierls per questo sistema sia legato intrinsecamente al problema del calcolo dei campi di Jacobi lungo una geodetica.

Nella terza parte vengono descritte due  realizzazione dello schema di quantizzazione algebrico per i campi bosonici. La prima sfrutta le parentesi di Peierls mentre la seconda interviene sui dati iniziali della dinamica di campo.
\newline
Il campo di Jacobi si presta ad essere quantizzato secondo entrambe le prescrizioni.
Confrontando le 2 forme simplettiche così ottenute si cerca di fornire nuovi tasselli per attribuire un'interpretazione geometrica al metodo originale di Peierls.


	\begin{Warning}
	INtro: paragrafo sul quaderno: "qual'è l'interesse che spinge a quantizzare questo sistema "Campo di Jacobi"?
	\end{Warning}
\fi










\end{document}