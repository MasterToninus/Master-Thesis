 \begin{abstract}
La prima parte della tesi è stata rivolta allo studio del framework matematico necessario per dare una formulazione rigorosa dei sistemi classici continui, punto di partenza di ogni schema di quantizzazione algebrica.
Nello specifico viene fatta una digressione sui Fibrati Topologici e viene sfruttata la definizione di fibrato liscio per presentare l'approccio geometrico alla meccanica classica sia per sistemi a gradi di libertà finiti che continui.

Nella seconda parte viene presentato l'algoritmo di Peierls che rappresenta una “ricetta” efficace per attribuire una struttura pre-simplettica allo spazio delle configurazioni dinamiche di un sistema qualunque.
Dalla ricerca bibliografica è evidente come questo strumento a partire dal suo esordio (nel 1952) fino ad oggi non abbia mai ricevuto particolare attenzione. Questo sembra dovuto soprattutto alla mancanza di una convincente interpretazione geometrica.
\newline
Per fare un passo verso la comprensione di questo oggetto viene studiato l’estremamente noto problema della geodetica vedendolo come un sistema campo.
Emerge sin da subito come il calcolo delle parentesi Peierls per questo sistema sia legato intrinsecamente al problema del calcolo dei campi di Jacobi lungo una geodetica.

Nella terza parte vengono descritte due  realizzazione dello schema di quantizzazione algebrico per i campi bosonici. La prima sfrutta le parentesi di Peierls mentre la seconda interviene sui dati iniziali della dinamica di campo.
\newline
Il campo di Jacobi si presta ad essere quantizzato secondo entrambe le prescrizioni.
Confrontando le 2 forme simplettiche così ottenute si cerca di fornire nuovi tasselli per attribuire un'interpretazione geometrica al metodo originale di Peierls.

\end{abstract}