%^^^^^^^^^^^^^^^^^^^^^^^^^^^^^^^^^^^^^^^^^^^^^^^^^^^^^^^^^^^^^^^^^^^^^^^^^^^^^^^
%		Panoramica del sistema geoderico 2D
%_______________________________________________________________________________

\documentclass{standalone}

\usepackage{amssymb}

\usepackage{tikz}
%\usetikzlibrary{...}
\usepackage{tikz}
\usepackage{pgfplots}
\pgfplotsset{compat=newest}
\usetikzlibrary{calc}

\usepackage[graphics,tightpage,active]{preview}
\PreviewEnvironment{tikzpicture}

%Common symbols
%Common math symbols

%Real Numbers
	\newcommand{\Real}{\mathbb{R}}
% Kinematic Configurations
	\newcommand{\Conf}{\mathrm{C}}
% Atlas
	\newcommand{\Atlas}{\mathcal{A}}

%Lagrangiana
	\newcommand{\Lagrangian}{\mathcal{L}}

%Solutions Space
	\newcommand{\Sol}{\mathrm{Sol}}

\begin{document}
    \begin{tikzpicture}
       \node (a) at (0,0)
         {
			\begin{tikzpicture}[scale=0.8]
				\begin{axis}[view={-30}{30},no markers,axis lines = middle,ticks=none ,
					ymax=15,ymin=-5,
					xmax=15,xmin=-5,
					zmax=15, zmin=-10]
						\addplot3+[domain=-0:30,samples=200,samples y=0,ultra thick](x-5,{3*sin(deg(1.2*x))},{3*cos(deg(1.2*x))+x/5});
				\end{axis}
			 \end{tikzpicture}
         };
        \node (b) at (8,0)
         {
			\begin{tikzpicture}[remember picture,scale=0.8]
				\begin{axis}[axis lines=none,clip=false]
					% Il riquadro di \Conf
					\addplot[color=black] coordinates {
						(8,6) (0,6) (0,0)(8,0)(8,6)
					}node [pos=1,pin={0:$\Gamma_0$},inner sep=0pt] {};
					
					% la retta di \Sol
					\addplot[color=red] coordinates {
						(0,2)	
						(8,2)
					}node [pos=1,pin={0:$\Sol$},inner sep=0pt] {};

					% Zero section
					\node(O)[label={270:{$0$}},circle,fill,inner sep=1pt] at (axis cs:1.25,2) {};
					% Fixed Solution
					\node[label={270:{$\vec{X}$}},circle,fill,inner sep=1pt] at (axis cs:5,2) {};

					% Osservabile associato alla soluzione
					\draw[-latex,color=black] (O) -- (4,4) node [pos=1,label={$\chi=E^{-1} \vec{X}$}]  {};  


				\end{axis}
			\end{tikzpicture}
         };
        \node (c) at (3.5,5.5) [ xscale=1.1, yscale=0.9]
         {
         \begin{tikzpicture}
         \begin{axis}[	axis x line=bottom, axis y line=left,
						ticks=none,
						xlabel={$x_1$},ylabel={$x_2$},
						xlabel style={below right},ylabel style={above left},
						xmin=-1, xmax=16, ymin=-0.8, ymax=3
						]
 
			% Geodesica di partenza
			\addplot[blue,domain=1:14,variable=\t] ({t},{sin(x^2)*x^(1/4)})		
				\foreach \p in {0,1,...,20}{						%punti per ottenere il vettore variazione
					node[sloped,inner sep=0cm,above,pos=\p/20,
					anchor=south west]
					(N \p){}
  				};	
  			% Geodesica Variata
  			\addplot[gray,domain=1:14,variable=\t] ({t+0.5*(1+sin(t*10))},{sin(x^2)*x^(1/4)+(1 +0.5*cos(t*10))*t/10+0.3})
				\foreach \p in {0,1,...,20}{						%punti per ottenere il vettore variazione
					node[	sloped,inner sep=0cm,above,pos=\p/20,
      					anchor=south west]
      				(M \p){}
  				};	
			%I Dati Iniziali
			\fill (N 0.south) circle [radius=2pt]node [label={225:{$\Sigma$}}]  {};  
			\draw[-latex,color=green] (N 0) -- (M 0) node [pos=0.5,label={180:{$X_0$}}]  {};    
			
			%La velocità iniziale la metto a mano in modo che si legga chiaramente 
			\draw[-latex,color=red] (N 0)  -- ($(N 1) + (0.8,-0.7)$) node [pos=0.5,label={270:{$V_0$}}]  {};
			\node[black] at (axis cs:6.5,2.4) {$\vec{X}$};
			\node [blue]  at (axis cs:6.5,0.8)   {$\gamma_0$};
		
		\end{axis}
		
		%Lo Jacobi Field
		\foreach \p in {1,2,...,20} {
			\draw[-latex,color=black] (N \p) -- (M \p);
    	}
		\end{tikzpicture}
         };
         
       %Mappe tra gli spazi componenti
        \node (Sol)  at (5.4,-0.75){};
        \node (Data) at (2.7,-0.75){};
              
       \path[->]		
       		(Data.north)    edge[bend left=60,above]  node        {$\SolMap$}  (Sol)
      		(Sol)    edge[bend left=60,below]  node        { \tiny $\left( \left. \cdot \right\vert_\Sigma , \left. D_t( \cdot ) \right\vert_\Sigma  \right)$}  (Data.south);

    \end{tikzpicture}


 

\end{document}