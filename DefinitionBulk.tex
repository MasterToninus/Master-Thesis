\documentclass[Main]{subfiles}

\begin{document}
	(sono ripetizioni inutili per la tesi, sono informazioni che si ritrovano ovunque... sono informazioni adatta al knowledge base)

	
	Recurring definitions in general Relativity (excluding the general smooth manifold prolegomena).

	\begin{definition}[Space-Time]
		A quadruple $(M, g, \mathfrak{o}, \mathfrak{t})$ such that:
		\begin{itemize}
			\item $(M,g)$ is a time-orientable n-dimensional manifold $(n>2)$
			\item $\mathfrak{o}$ is a choice of orientation
			\item $\mathfrak{t}$ is a choice of time-orientation
		\end{itemize}
	\end{definition}

	\begin{definition}[Lorentzian Manifold]
		A pair $(M, g)$ such that:
		\begin{itemize}
			\item $M$ is a n-dimensional $(n\geq2)$, Hausdorff, second countable, connected, orientable smooth manifold.
			\item $g$ is a Lorentzian metric.
		\end{itemize}
	\end{definition}
			
	\begin{definition}[Metric]
		A function on the bundle product of $TM$ with itself: $$g: TM \times_M TM \rightarrow \Real$$ such that the restriction on each fiber $$g_p: T_pM \times T_pM \rightarrow \Real $$ is a non-degenerate bilinear form.
	\end{definition}
	
	\begin{notationfix}
		 \begin{itemize}
		 	\item \emph{Riemman} if the sign of $g$ is positive definite, \emph{Pseudo-Riemman} otherwise.
		 	\item \emph{Lorentzian} if the signature is $(+, -, \ldots,- )$ or equivalently $(-,+,\ldots,+)$.
		 \end{itemize}
	\end{notationfix}

	\begin{observation}[Causal Structure]
		If a smooth manifold is endowed with a Lorentzian manifold of signature $(+, -, \ldots, -)$ then the tangent vectors at each point in the manifold can be classed into three different types. 
		\begin{notationfix}
			$\forall p \in M, \quad \forall X \in T_pM$, the vector is:
			\begin{itemize}
				\item \emph{time-like} if $g(X,X)>0$.
				\item \emph{light-like} if $g(X,X)=0$.
				\item \emph{space-like} if $g(X,X)<0$.
			\end{itemize}
		\end{notationfix}
	\end{observation}

	\begin{observation}[Local Time Orientability]
		$\forall p\in M$ the timelike tangent vectors in $p$ can be divided into two equivalence classes taking
		\begin{displaymath}
			X \sim Y \; \textrm{iff} \; g(X,Y)>0 \qquad \forall X,Y \in T^\textrm{time-like}_pM:
		\end{displaymath}
		We can (arbitrarily) call one of these equivalence classes "future-directed" and call the other "past-directed". Physically this designation of the two classes of future- and past-directed timelike vectors corresponds to a choice of an arrow of time at the point. The future- and past-directed designations can be extended to null vectors at a point by continuity.
	\end{observation}
	
	\begin{definition}[Time-orientation]
		A global tangent vector field  $\mathfrak{t}\in \Gamma^\infty(TM)$ over the Lorenzian manifold $M$ such that:
		\begin{itemize}
			\item $\supp(\mathfrak{t}) = M$
			\item $\mathfrak{t}(p)$ is time-like $\forall p \in M$.
		\end{itemize}
	\end{definition}
	\begin{observation}
		The fixing of a time-orientation is equivalent to a consistent smooth choice of a local time-direction.
	\end{observation}	
	
	\begin{definition}[Time-Orientable Lorentzian Manifold]
		A Lorentzian Manifold $(M,g)$ such that exist at least one time-orientation $\mathfrak{t}\in \Gamma^\infty(TM)$.
	\end{definition}

	\begin{notationfix}
		Consider a piece-wise smooth curve $\gamma: \Real\supset I \rightarrow M$ is called:
		\begin{itemize}
			\item \emph{time-like} (resp. light-like, space-like) iff $\dot{\gamma}(p)$ is time-like (resp. light-like, space-like) $\forall p \in M$.
			\item \emph{causal} iff $\dot{\gamma}(p)$ is nowhere spacelike.
			\item \emph{future directed} (resp. past directed) iff is causal and  $\dot{\gamma}(p)$ is future (resp. past) directed $\forall p \in M$.
		\end{itemize}
	\end{notationfix}

	\begin{definition}[Chronological $\substack{\textrm{ future}\\ \textrm{past } }$ of a point]
		Are two subset related to the generic point $p	\in M$:
		\begin{displaymath}
			\mathbf{I}_M^\pm(p) \coloneqq \big\{ q \in M \big\vert \; \exists \gamma \in C^\infty\big((0,1), M\big)\;  \textrm{\footnotesize time-like } \substack{\textrm{future}\\ \textrm{past} } -\textrm{\footnotesize directed }:\; \gamma(0)=p,\; \gamma(1)=q  \big\}
		\end{displaymath}
	\end{definition}
	
	\begin{definition}[Causal $\substack{\textrm{ future}\\ \textrm{past } } $ of a point]
		Are two subset related to the generic point $p	\in M$:
		\begin{displaymath}
			\mathbf{J}_M^\pm(p) \coloneqq \big\{ q \in M \big\vert \; \exists \gamma \in C^\infty\big((0,1), M\big)\; \textrm{\footnotesize causal } \substack{\textrm{future}\\ \textrm{past} } -\textrm{\footnotesize directed }:\; \gamma(0)=p,\; \gamma(1)=q  \big\}
		\end{displaymath}		
	\end{definition}

	\begin{notationfix}
		Former concept can be naturally extended to subset $A \subset M$:
			\begin{itemize}
				\item $\mathbf{I}_M^\pm(A) = \bigcup_{p\in A} \mathbf{I}_M^\pm(p) $
				\item $\mathbf{J}_M^\pm(A) = \bigcup_{p\in A} \mathbf{J}_M^\pm(p) $
			\end{itemize}
	\end{notationfix}

	\begin{definition}[Achronal Set]
		Subset $\Sigma \subset M$ such that every inextensible timelike curve intersect $\Sigma$ at most once.
	\end{definition}

	\begin{definition}[$\substack{\textrm{ future}\\ \textrm{past } } $ Domain of dependence of an Achronal set]
		The two subset related to the generic achornal set $\Sigma \subset M$:
		\begin{displaymath}		
			\mathbf{D}_M^\pm(\Sigma) \coloneqq \big\{ q \in M \big\vert \; \forall \gamma \substack{\textrm{ past}\\ \textrm{ future} }\textrm{\footnotesize inextensible causal curve passing through }q : \; \gamma(I) \cap \Sigma \neq \emptyset  \big\}
		\end{displaymath}		
	\end{definition}

	\begin{notationfix}
		$\mathbf{D}_M(\Sigma)  \coloneq \mathbf{D}_M^+(\Sigma) \cup \mathbf{D}_M^-(\Sigma)$ is called \emph{total domain of dependence}.
	\end{notationfix}

	\begin{definition}[Cauchy Surface]
		Is a subset $\Sigma \subset M$ such that:
		\begin{itemize}
			\item closed
			\item achronal
			\item $\mathbf{D}_M(\Sigma) \equiv M$
		\end{itemize}
	\end{definition}


\newpage
Basic Definition in L.P.D.O. on smooth vector sections.
\\
Consider $F=F(M,\pi,V), F'=F'(M,\pi',V')$ two linear vector bundle over $M$ with different typical fiber
	\begin{definition}[Linear Partial Differential operator \footnotesize( of order at most $s\in \Natural_0$)]
		Linear map $L:\Gamma(F)\rightarrow \Gamma(F')$ such that:
		\\
		$\forall p \in M$ exists:
		\begin{itemize}
			\item $(U, \phi)$ local chart on $M$.
			\item $(U, \chi)$ local trivialization of $F$
			\item $(U, \chi')$ local trivialization of $F'$
		\end{itemize}
		for which:
		\begin{displaymath}
			L \big(\sigma \big\vert_U\big) = \sum_{\vert \alpha \vert \leq s} A_\alpha \partial^\alpha \sigma \qquad \forall \sigma \in \Gamma(M)
		\end{displaymath}
	\end{definition}

	\begin{remark}
	(multi-index notation)
	\\
	A multi-index is a natural valued finite dimensional vector $\alpha = ( \alpha_0, \ldots, \alpha_n-1) \in \Natural^n_0$ with $n<\infty$.
	\\
	On $\Real^n$ a general differential operator can be identified by a multi-index:
	\begin{displaymath}
		\partial^\alpha = \prod_{\mu = 0}^{n-1} \partial_\mu ^{\alpha_\mu}
	\end{displaymath}
	(Until the Schwartz theorem holds, the order of derivation is irrelevant.)
	\\
	The order of the multi-index is defined as:
	\begin{displaymath}
		\vert \alpha \vert \coloneqq \sum_{\mu=0}^{n-1} \alpha_\mu
	\end{displaymath}
	\end{remark}

	?????????????????????
	\begin{proposition}[Existence and uniqueness for the Cauchy Problem]
		\begin{hypothesis}
			\begin{itemize}
				\item $\mathbf{M} = (M,g,\mathfrak{o},\mathfrak{t}) $a globally hyperbolic space-time.
				\item $\Sigma \subset M$ a spacelike cauchy surface with future-pointing unit normal vector field $\vec{n}$.
				\item 
			\end{itemize}
		\end{hypothesis}
	\begin{thesis}

	\end{thesis}
	\end{proposition}
	\begin{observation}
	"Green-hyperbolic operators are not necessarily hyperbolic in any PDE-sense and that they cannot be characterized in general by well-posedness of a Cauchy problem.	" \cite{Terlaky2010} \cite{Bar2010}
	\\
	However the existence and uniqueness can be proved for the large class of the \emph{Normally-Hyperbolic Operators}.
	
	\end{observation}


\end{document}