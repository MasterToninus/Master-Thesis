\documentclass[11pt]{article}
\pdfminorversion=4

\title{Approccio Geometrico alle parentesi di Peierls nella quantizzazione algebrica del campo Geodetico. }
\author{Antonio Michele Miti}
\date{}

\usepackage[a4paper,top=4.25cm,bottom=3.75cm,left=3.25cm,right=3cm]{geometry}
\usepackage{pdfpages} %aggiuntaPdf
\usepackage{amssymb}

\usepackage[italian]{babel} % Explicitly load the babel package to stop an error occurring on some LaTeX installations
\usepackage[utf8x]{inputenc} %lettere accentate

\usepackage[a-1b]{pdfx}% il pdf archiviabile richiesto da unimi


\begin{document}
\maketitle

\begin{center}
\begin{tabular}{l c l}
Relatore & : & Claudio Dappiaggi \\
Relatore Interno & : & Livio Pizzocchero \\
\end{tabular}

\end{center}


\section{Motivazione della Tesi}


%Cosa vogliamo Fare

Le parentesi di Peierls sono un elemento cruciale dello schema di quantizzazione algebrica, forniscono una “ricetta” efficace per attribuire una struttura pre-simplettica allo spazio delle configurazioni dinamiche per una generica teoria di campo su spazio tempo Globalmente Iperbolico.
Dalla ricerca bibliografica è evidente come questo strumento, a partire dal suo esordio nel 1952 fino ad oggi, non abbia mai ricevuto particolare attenzione. Questo sembra dovuto soprattutto alla mancanza di una convincente interpretazione geometrica che ha avuto l'effetto di limitarne la diffusione.

Per fare un passo verso la comprensione di questo oggetto viene studiato l’estremamente noto problema della geodetica riguardandolo come un sistema campo.
L'esempio è notevole da due punti di vista:
\begin{enumerate}
	\item in quanto sistema a gradi di libertà finiti:
	\begin{itemize}
		\item Le configurazioni cinematiche sono curve parametrizzate su una varietà Riemmaniana. Il sistema è un campo molto semplice in cui la varietà base è banalmente $\mathbb{R}$ quindi rappresenta un esempio complementare al solito campo reale scalare.
		\item Secondo lo schema di quantizzazione per dati iniziali, i dati di Cauchy per sistemi di 	questo tipo sono semplicemente coppie di vettori finito-dimensionali. In questo caso la forma simplettica associata è unica e si può provare che corrisponde a quella costrutita tramite il metodo, completamente differente, di Peierls.
	\end{itemize}

	\item In quanto campo geodetico:
	\begin{itemize}
		\item Il sistema geodetico è non-lineare. L'applicazione dello schema di quantizzazione algebrica e dell'algoritmo di Peierls richiede di passare per la sua linearizzazione, ovvero per il  \emph{Campo di Jacobi}. 
		Anche questo è un oggetto matematico ampiamente studiato dal punto di vista della geometria differenziale ma molto raramente è stato trattato come un sistema di tipo campo. 
		\item L'operatore delle equazioni di Jacobi è \emph{Normalmente Iperbolico}, il che permette di affrontare la quantizzazione del sistema secondo due schemi diversi: quello che sfrutta le parentesi di Peierls, in quanto tale operatore è Green Iperbolico e formalmente autoggiunto, e quello per dati iniziali, in quanto Iperbolico nel senso delle EDP.
	\end{itemize}
\end{enumerate}

Siccome il campo di Jacobi si presta ad essere quantizzato  sia secondo lo schema di Peierls che secondo lo schema per Dati Iniziali, il confronto con le 2 forme simplettiche così ottenute permette di attribuire un'interpretazione geometrica al metodo originale di Peierls.

\section{Metodi di indagine}
La prima parte della tesi è stata rivolta allo studio del framework matematico necessario per dare una formulazione rigorosa dei sistemi classici continui, punto di partenza di ogni schema di quantizzazione algebrica.
Nello specifico è stata fatta una digressione sui Fibrati Topologici con l'obbiettivo di sfruttare la definizione di fibrato liscio per presentare un approccio geometrico astratto alla meccanica Lagrangiana classica che permetta di trattare in modo unificato sia i sistemi a gradi di libertà finiti che quelli a gradi di libertà continui.

La seconda parte del lavoro è stata votata alla ricerca bibliografica riguardo le partentesi di Peierls.
Siamo partiti dall'articolo originale di Peierls \cite{Peierls} e abbiamo confontato questo approccio con quelli più moderni, essenzialmente proposti da DeWitt\cite{DeWitt} e da Forger e Romero \cite{Forger}.
Mentre in questi ultimi due casi la forma simplettica viene essenzialmente postulata su delle opportune classi di \emph{osservabili classici} il metodo di Peierls permette di dare una definizione più generale su tutti i funzionali lagrangiani attraverso la definizione dell' \emph{effetto di una perturbazione lagrangiana su una soluzione imperturbata}.

Per poter proseguire nell'indagine è stato necessario un ulteriore studio riguardante gli schemi di quantizzazione algebrica bosonica, nello specifico sono stati rivisti lo schema con le parentesi di Peierls e quello attraverso i dati iniziali.

In tutto il lavoro è stato seguito un approccio deduttivo. Abbiamo dapprima presentato i sistemi classici in astratto, poi gli schemi di quantizzazione e solo alla fine abbiamo realizzato tutte queste strutture nel caso del campo geodetico.

\section{Risultati ottenuti}

Il confronto con le numerose fonti analizzate ci ha permesso di sintetizzare, in un linguaggio più moderno e rigoroso,  l'algoritmo  di costruzione  delle parantesi proposto da Peierls per una classe di sistemi abbastanza generale: i sistemi Lagrangiani con operatore del moto Green-Iperbolico e con Operatori di Green unici.

Abbiamo sottolineato come da questa costruzione si possano ottenere in modo pressochè immediato la struttura simplettica di pre-quantizzazione che usualmente viene assunta come un postulato negli schemi di quantizzazione algebrica.

Abbiamo realizzato le parentesi di Peierls e due schemi di quantizzazione algebrica per l'esempio insolito del \emph{Moto Geodetico} e fatto emergere un legame tra le forme simplettiche ottenute tramite queste due differenti costruzioni.

\begin{thebibliography}{}

  \bibitem{Peierls} Peierls, R. E. (1952). The Commutation Laws of Relativistic Field Theory. Proceedings of the Royal Society A: Mathematical, Physical and Engineering Sciences, 214(1117), 143–157. 
	
	\bibitem{DeWitt} Dewitt, B. (1999). The Peierls Bracket. 	In Quantum Field Theory: Perspective and Prospective SE  - 5 (Vol. 530, pp. 111–136). Springer Netherlands. 
	
	\bibitem{Forger}  Forger, M., Romero, S. V. (2005). Physics Covariant Poisson Brackets in Geometric Field Theory, 410, 375–410.
	
	
  \end{thebibliography}

\end{document}


\section{Introduzione}
%E' ormai di pubblico dominio il riconoscimento della teoria quantistica dei campi (QFT) di essere la teoria più di successo della storia..
Quantum Field Theory (QFT)  is the synthesis of Quantum Mechanics and Special Relativity and it is the general framework for the description of the physics of relativistic quantum systems.
Its most direct applications, quantum electrodynamics and the standard model of particles, are both been experimentally verified to an outstanding degree of precision and they allowed us to have an almost fully satisfactory and unified description of the electro-weak forces.

In any case it is by no mean a definitive theory. It is clear that the intrinsic quantum description of elementary particles clashes with the the structure of a deterministic theory, as general relativity. 
It is almost unanimously accepted that a quantum theory of gravity is needed in order to reconcile general relativity with the principles of quantum mechanics. Yet, despite countless efforts, a quantum theory of the gravitational interaction remains an open problem
% Although a quantum theory of gravity is needed in order to reconcile general relativity with the principles of quantum mechanics. 

%l'idea per giungere alla AQFT è per due strade, la prima è quella della rigorizzazione della qed (aximoatic) la seconda è che si tratta del formalismo più generale per estendere la QFT da spazi tempi di minkoski a spazi tempi curvi opportuni.
While quantum field theory has been a well established topic for the past 50 years, the quest to finding of a \emph{theory of everything} has often lead the community to neglect the role of two important aspects concerning the QFT.\\
The first is the existence of an intermediate regime, going under the name of \emph{QFT in curved background}, which is expected to provide an accurate description of quantum phenomena in a regions where the effects of curved spacetime may be significant, but effects of quantum gravity itself may be neglected. 
Many successful application of this idea can be found in context of the theory of cosmological inflation or black holes thermodynamic.\\
The second is the construction of a mathematically rigorous description of quantum fields, in particular of their non-perturbative aspects,  based on a sound and shared set of first principles. In other words an \emph{axiomatic foundation of QFT}.

At the moment, \emph{algebraic quantum field theory} (AQFT) is %\ifToninus proven to be the most promising\fi 
a way to complete the picture of the quantum theory of fields regarding the above two aspects.
Its aim is to reach a general and mathematical rigorous description of the foundations of quantum fields on a  sufficiently large class of curved, but fixed, backgrounds.

The algebraic approach, as the majority of contemporary quantum field theory, is developed as a quantization of classical fields.
%Da questo deve riconusciuto che una formalizzazione completa dei fondamenti della QFT non si può esimere da uno costruzione matematicamente precisa dei sistemi campi classici.
%From that should be clear that 
As a starting point, a mathematically rigorous \emph{classical field theory}  is thus a necessary step towards the understanding of the foundations of  the theory.
It has to be noted that classical fields such the force fields of analytical mechanics or the stress tensor in fluid dynamics are not of much interest  insofar QFT is concerned. % più che stress tensor intendevo i CAMPI MATERIALI della fisica dei continui
What is essential to determine is a proper definition of the (pre-quantum) classical analogue of the "fundamental" fields.
%Nell'accezione che riguarda i fondamenti di qft non sono tanto  di interesse i campi classici intesi come campi di forza della meccanica analitica o i tensori definenti i continui materiali, ma bensi potremo dire la versione pre quantistica dei campi fondamentali.
Particularly, it is crucial the identification of the field-theoretic equivalent of the geometric structures underlying the canonical formalism of classical mechanics, namely the \emph{phase space} and an \emph{algebra of classical observables}. 
%In particolar modo è cruciale l'identificazione dell'equivalente field-theoretic delle strutture geometriche proprie del formalismo canonico della meccanica classica, namely the \emph{phase space} and \emph{classical observables algebra}. 
From an abstract point of view  the first is a symplectic manifold, namely a smooth manifold endowed with a non-degenerate 2-form, while the second is a Poisson algebra constituted by functionals on the phase space.

It is important to emphasize that the algebraic quantization is not a unique and well-defined algorithm that reads in a system of classical mechanics and returns a corresponding quantum mechanical system.
Rather it should be seen as a \emph{quantization scheme} which can be realized by several specific procedures.
%La quantizzazione algebrica non si tratta di un unico ben delineato algoritmo that reads in a system of classical mechanics and returns a corresponding system of quantum mechanics, ma bensì di uno schema di quantizzazione che può essere realizzato da differenti procedure specifiche.
All of them are based on a set set of first principles, essentially proposed by Dimock \cite{Dimock1980} as an extension of the axioms of Haag and Kastler formulated on Minkowski spacetime, by prescribing the mathematical structure of the quantum observables algebra.
% che prescrivono la struttura che deve soddisfare l'algebra degli osservabili .
What in which such realizations of the algebraic scheme essentially differ, is in the different identification of the suitable symplectic manifold  to be associated with the pre-quantum version of the field under examination.
%Ciò in cui essenzialmente differiscono queste "realizzazioni" dello schema algebrico,  è nella diversa identificazione delle giusta varietà simplettica da associare alla versione pre quantistica del campo.

The most common way to building these structures requires an explicit choice of \emph{Cauchy surface} in the underlying spacetime. 	
This leads to a  realization of the algebraic quantization known as \emph{quantization via initial data}.

	
Of much greater interest is the , fully covariant, construction based on the so-called \emph{covariant phase space} and on the \emph{Peierls brackets}.
	The first is defined as the \emph{space of dynamical configurations}, \textit{i.e.}  the infinite-dimensional space of solutions of the equations of motion, while the second is a particular choice of Poisson brackets.% attributable to a field system.
	
	The construction of such Poisson brackets is achieved following what we call the \emph{Peierls' algorithm}, a procedure originally proposed by Rudolf E. Peierls in a seminal paper dated 1952 \cite{Peierls1952}. 
	This is an effective, but rather convoluted, "recipe"  to prescribe a pre-symplectic structure on the space of dynamical configurations. 
	Browsing through the literature, it is clear that the Peierls' construction never received particular attention since its formulation.
This can be ascribed mainly to the lack of a convincing geometric interpretation
which had the effect of limiting its reception often relegating its role to that of a mere  "mathematical trick".

L'algoritmo sviluppato da Peierls costituisce un elemento cruciale all'interno dello schema di quantizzazione algebrica. Questo fornisce una “ricetta” efficace per attribuire una struttura pre-simplettica allo spazio delle configurazioni dinamiche per un generico sistema campo su spazio tempo Globalmente Iperbolico.



\section{Motivazione della Tesi}


%Cosa vogliamo Fare
Lo scopo di questa tesi è quello di analizzare la procedura orginale di Peierls in ogni suo aspetto, cercando di adattare ogni singolo passo ad un linguaggio matematico più moderno e rigorso.