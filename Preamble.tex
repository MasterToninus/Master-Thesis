%Per le Figure
\usepackage{graphicx}

%Hint CD: per lo spelling corretto negli a capo
\usepackage[italian,english]{babel}
\usepackage[utf8]{inputenc}

%simboli matematici strani quali unione disgiunta
\usepackage{amssymb}

%Scrivere Sotto i simboli
\usepackage{amsmath}

%Diagrammi Commutativi
\usepackage{tikz}
\usetikzlibrary{matrix}

%Pacchetto standalone per implementare in modo ordinato le figure tikz
\usepackage[subpreambles]{standalone}

%Il simbolo di Identità
\usepackage{dsfont}

%Per riflettere i simboli...
\usepackage{graphicx}


%link iNTERNET
\usepackage{hyperref}

%Enumerate with letters
\usepackage{enumerate}

%Slash over letter
\usepackage{cancel}

%Usare bibiliografia biblatex
%\usepackage[style=numeric,natbib=true]{biblatex}
%\addbibresource{ThesisBiblio.bib}



%Danger sign
\usepackage{fourier}

%:=
\usepackage{mathtools}

%http://tex.stackexchange.com/questions/8625/force-figure-placement-in-text
\usepackage{caption}

%inline fraction
%\usepackage{nicefrac}

%subsection numbering
 \setcounter{tocdepth}{3} % if you want all the levels in your table of contents

%Common symbols
%Common math symbols
	%Number fields
		\newcommand{\Real}{\mathbb{R}}
		\newcommand{\Natural}{\mathbb{N}}
		\newcommand{\Relative}{\mathbb{Z}}
		\newcommand{\Rational}{\mathbb{Q}}
		\newcommand{\Complex}{\mathbb{C}}
	
%equality lingo
	%must be equal
		\newcommand{\mbeq}{\overset{!}{=}} 

% function
	%Domain
		\newcommand{\dom}{\mathrm{dom}}
	%Range
		\newcommand{\ran}{\mathrm{ran}}
	

% Set Theory
	% Power set (insieme delle parti
		\newcommand{\PowerSet}{\mathcal{P}}

%Differential Geometry
	% Atlas
		\newcommand{\Atlas}{\mathcal{A}}
	%support
		\newcommand{\supp}{\textrm{supp}}

	
	
%Category Theory
	%Mor set http://ncatlab.org/nlab/show/morphism
%		\newcommand{\hom}{\textrm{hom}}

%Geometric Lagrangian Mechanics
	% Kinematic Configurations
		\newcommand{\Conf}{\mathtt{C}}
	%Solutions Space
		\newcommand{\Sol}{\mathtt{Sol}}
	%Lagrangian class
		\newcommand{\Lag}{\mathsf{Lag}}
	%Lagrangiana
		\newcommand{\Lagrangian}{\mathcal{L}}
	%Data
		\newcommand{\Data}{\mathsf{Data}}
	%unique solution map
		\newcommand{\SolMap}{\mathbf{s}}
	%Classical Observables
		\newcommand{\Obs}{\mathcal{E}}	
	%Phase Space
		\newcommand{\Phase}{\mathcal{M}}	

		\
		
%Peierls (per non sbagliare più)
		\newcommand{\Pei}{Peierls}

%Accented Letters  (lot of stuff... got problems with dieresys)
\usepackage[utf8]{inputenc}


%Temporaneo, Aggiunta della mia classe teorem... Deve diventare un pacchetto!
\ifToninus
\input{../Latex-Theorem/TheoremTemplateToninus.tex}
\else
\input{../Latex-Theorem/TheoremTemplateToninus_plain.tex}
\fi

%Fancyhdr http://tex.stackexchange.com/questions/118647/line-under-chapter-name-at-top-of-the-page
\usepackage{fancyhdr}
\pagestyle{fancy}
\fancyhf{}
\fancyhead[EL]{\nouppercase\leftmark}
\fancyhead[OR]{\nouppercase\rightmark}
\cfoot{\thepage}
\fancyhead[ER,OL]{\thepage}


%Fancy Chapter Title
\usepackage[Lenny]{fncychap}

%Formato di Pdf archiviabile espressamente espresso da Unimi
\usepackage[a-1b]{pdfx}

% Attaccare il frontespizio come pdf a parte
\usepackage{pdfpages} 

% Comprimere le liste (per farle entrare bene nelle definizioni
\usepackage{enumitem}

\newenvironment{compactitemize}
	{ 	
		\begin{itemize}[nosep,itemsep=0.5mm,topsep=0.7mm]%nosep
			\renewcommand\labelitemi{$\cdot$}
	}
	{ \end{itemize} }
	
\newenvironment{compactdisplaymath}
	{ 	
		\vspace{-3mm}
		\begin{displaymath}
	}
	{ \end{displaymath}  }


% Blank Page http://tex.stackexchange.com/questions/36880/insert-a-blank-page-after-current-page
\usepackage{afterpage}
\usepackage{emptypage}

\newcommand\blankpage{%
    \null
    \thispagestyle{empty}%
    \addtocounter{page}{-1}%
    \newpage}
 
\newcommand\forcenewline{\hfill \break \vspace{-3mm}}
    

 

