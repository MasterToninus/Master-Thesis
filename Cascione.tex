%\documentclass[a4paper,12pt]{scrartcl}
\documentclass[Main]{subfiles}


\title{Pezzi temporanei e parti eliminate dalla tesi. }
\author{Toninus}
%%Per le Figure
\usepackage[english]{babel}
\usepackage{graphicx}

%simboli matematici strani quali unione disgiunta
\usepackage{amssymb}

%Scrivere Sotto i simboli
\usepackage{amsmath}

%Diagrammi Commutativi
\usepackage{tikz}
\usetikzlibrary{matrix}

%Il simbolo di Identità
\usepackage{dsfont}

%Per riflettere i simboli...
\usepackage{graphicx}


%link iNTERNET
\usepackage{hyperref}

%Enumerate with letters
\usepackage{enumerate}

%Slash over letter
\usepackage{cancel}

%Usare bibiliografia bibtex
%\bibliographystyle{plain}

%Danger sign
\usepackage{fourier}

%:=
\usepackage{mathtools}

%http://tex.stackexchange.com/questions/8625/force-figure-placement-in-text
\usepackage{caption}

%inline fraction
%\usepackage{nicefrac}

%subsection numbering
 \setcounter{tocdepth}{3} % if you want all the levels in your table of contents

%Common symbols
%Common math symbols

%Real Numbers
	\newcommand{\Real}{\mathbb{R}}
% Kinematic Configurations
	\newcommand{\Conf}{\mathrm{C}}
% Atlas
	\newcommand{\Atlas}{\mathcal{A}}

%Lagrangiana
	\newcommand{\Lagrangian}{\mathcal{L}}

%Solutions Space
	\newcommand{\Sol}{\mathrm{Sol}}

%Accented Letters
\usepackage[utf8]{inputenc}

%Temporaneo, Aggiunta della mia classe teorem... Deve diventare un pacchetto!
%\input{../Latex-Theorem/TheoremTemplateToninus_plain.tex}
\input{../Latex-Theorem/TheoremTemplateToninus.tex}


%Fancyhdr http://tex.stackexchange.com/questions/118647/line-under-chapter-name-at-top-of-the-page
\usepackage{fancyhdr}
\pagestyle{fancy}
\fancyhf{}
\fancyhead[EL]{\nouppercase\leftmark}
\fancyhead[OR]{\nouppercase\rightmark}
\cfoot{\thepage}
\fancyhead[ER,OL]{\thepage}


\begin{document}


%  Titolo
\maketitle
\begin{abstract}
	Sono un accumulatore.\\
	Tutti i mezzi testi, mezze intuizioni che non ho tradotto o a cui non ho trovato posto nella tesi le metto qui dentro.\\
\end{abstract}
%  Indice
\tableofcontents

%%%%%%%%%%%%%%%%%%%%%%%%%%%%%%%%%%%%%%%%%%%%%%%%%%%%%%%%%%%%%%%%%%%
\chapter{Prereq Mate}

	\begin{Warning}
	(Primer)\\
	This review is organized as follows: In Section 2 we discuss the key geometrical concepts which lie
at the heart of the construction of quantum field theory on curved backgrounds, particularly the notion of globally hyperbolic spacetimes. In Section 3, we focus instead on the classical description of linear free field theories, whose dynamics is ruled by hyperbolic partial differential equations on vector bundles. In particular we will be interested in those equations which admit advanced and retarded Green operators.
	\end{Warning}	
	\begin{Warning}
		Stile: Intro lapiadaria ai 3 argomenti ( bundle cinematica di campo, Glob iper stage per descrivere dinamica di tipo propagativo, Operatori tipo onda). Poi smitragliata di definizioni come faceva Penati.
	\end{Warning}	
	



	\section{Fiber Bundles}
	\begin{Warning}
	(Primer)\\
	As a starting point and for the sake of completeness, we recall the definitions of a vector bundle and of
its sections. The latter play a distinguished role since sections represent the natural mathematical object to associate to the physical idea of a classical field. For a detailed discussion of these topics we refer the reader
	\end{Warning}
	\begin{Warning}
	(Advances)
	As a starting point we introduce the building block of any classical and quantum field theory:
	\end{Warning}
	
	
	
		Cosa Serve Dire?
			\subsection{Formal Definition}
			Parto dalla definizione degli smooth.\\
			per la tesi interessano solo i lisci e i vettoriali.\\
			trivialization charts\\
			bundlemorphism\\
			\begin{Warning}
				Forse è necessario parlare del cartesian product. così posso parlare dei tensor bundle e quindi definire il prodotto scalare tra fibrati come una mappa sul primo tensor product?
				vedi def 9 di advances
				
			\end{Warning}			
			
			\subsection{Sections}
			Definizione\\
			
Formal definition ed esempi smooth e bundle (tutto ciò che 

			\subsection{Jet Bundles}
				\url{https://en.wikipedia.org/wiki/Jet_bundle}
				


%%%%%%%%%%%%%%%%%%%%%%%%%%%%%%%%%%%%%%%%%%%%%%%%%%%%%%%%%
	\section{Globally Hyperbolic SpaceTime}
		\begin{Warning}
		(ADVANCEs)\\
		Goal of this section is to introduce all geometric concepts and tools which are necessary to discuss both the classical dynamics and the quantization of a free quantum field on a curved background. We assume that the reader is familiar with the basic notions of differential geometry and, to a minor extent, of general relativity.
		\end{Warning}
			
		ricordare di parlare di :
		\\ Sezioni con supporto time, past , future, space - compact (fare schemino come in advances)

			\begin{Warning}
				Mettere solo le definizioni che uso prese dagli articoli di review delle Fonti
			\end{Warning}	
			Appunti che mi ero preso scrivendo il secono capitolo:
					
			This condition is strictly connected to the dynamic behaviour of the system.
	
		Glon iperbolic determina la fogliazione dello spazio tempo per superfici di cauchy
		
		questo da la possibilità della buona posizione dei problemi di cauchy.. fisicamente è la condizione minima per definire i dati iniziali dell'evoluzione dinamica.
		definisco data...			
		
		\begin{Warning}
			Far notare che minkowski e tanti esempi importanti sono GH
		\end{Warning}
		
		\begin{observation}
		(che serve dopo) lo spazio $R$ è banalmente iperbolico in quanto tutti i punti posso essere visti come superfici di cauchy.
		\end{observation}
		
	\section{Green Hyperbolic Operators}
		\begin{Warning}
		Rapporto con la condizione sugli operatori...		
		
	No!		La definizione di green hyperbolicity non garantisce invece l'esistenza e unicità del problema di cauchy associata
		
		e non solo, anche l'esistenza degli operatori di green associati che sono ingrediente fondamentale della costruzione di peierls

		M è glob iper e P è green iper per tener conto del comporatamento propagativo
		definire sup cauchy
		definire s-t iperbolico (solo la caratterizzazione di ammetre una sup di cauchy)
		definire op green iperbolico su spazio tempo iperbolico (cioè ha delle green ope)
		Propr di buona definizione esistenza e unicita della soluzione
		
		Di particolare ricorrenza fisica sono gli operatori normally iperbolic
		espressione in coordinate
		esempio K-g!
		\end{Warning}		
			\begin{Warning}
				Mettere solo le definizioni che uso prese dagli articoli di review delle Fonti
		\end{Warning}	
					\begin{Warning}
				Pensavo di utilizzare la definizione di Green hyperbolic data da Bar che si avvale del concetto di formally dual (che non richiede la presenza del pairing) invece di quella usata in Advances AQFT che richiede solo che ammetta almeno un $G^\pm$  per poi dimostrare tramite teorema che se è anche autoaggiunto vale l'unicità. Si tratta solo di una piccola sfumatura.. Deve essere chiarito che in tutto ciò che faccio interessano che $$\forall P \, \exists1!G^\pm$$.
				Che poi questa condizione derivi da GH secondo bar o Gh secondo dap+selfadj è una di quelle questioni propriamente matematiche che poco interessa ai fisici della commissione.
			\end{Warning}			
			
			\begin{Warning}
			Devo richiedere che il green operator sia unico? sia negli schemi di quantizzazione che nella definizione di peierls faccio largo uso dell'unicità. 
			Per provare questa unicità si passa per la definizione di una forma bilineare che permette di parlare di aggiunto formale e quindi avvalersi del teorema.
			\end{Warning}
			
		    Green-hyperbolic operators are not necessarily hyperbolic in any PDE-sense and that they cannot be characterized in general by well-posedness of a Cauchy problem. \cite{Terlaky2010} \cite{Bar2010}
		




%\/\/\/\/\/\/\/\/\/\/\/\/\/\/\/\/\/\/\/\/\/\/\/\/\/\/\/\/\/\/\/\/\/\/\/\/\/\/\/\/\/\/\/\/\/\/\/\/\/\/\/\/\/\/\/\/\/\/\/\/\/\/\/\/\/\/\/\/\/\/\/\/
%\/\/\/\/\/\/\/\/\/\/\/\/\/\/\/\/\/\/\/\/\/\/\/\/\/\/\/\/\/\/\/\/\/\/\/\/\/\/\/\/\/\/\/\/\/\/\/\/\/\/\/\/\/\/\/\/\/\/\/\/\/\/\/\/\/\/\/\/\/\/\/\/
\chapter{Lagrangian systems e Pierls}
	\section{Presentazione in Astratto}
		Quando parlo della cinematica mi piacerebbe dare indicazioni sulla struttura matematica dello spazio delle configurazioni cinematiche:
			\begin{enumerate}
				\item costituisce una frechet manifold ( gli unici risultati che ho trovato sono quelli di Palais di "non linear global analysis"
				\item le curve parametrizzate sono le variazioni
				\item classi di equivalenza definiscono delle variazioni infinitesime che costituiscono lo spazio tangente allo spazio delle configurazioni cinematiche
				\item questo spazio tangente è isomorfo allo spazio delle sezioni del pullback rispetto alla sezione $\phi\in C$ del verical bundle (vedere forger romero)
				\item il problema dell'atlante e della rappresentazione delle sezioni in carta locale ( da scegliere sia sul total space E che sul base space M)
			\end{enumerate}
		Dovrei fare riferimento al teorema di Ostrowsky per giustificare il fatto che consideriamo solo il primo ordine. le langrangiana con termini cinetici esotici sono instabili ( nel senzo che non ammetto come soluzioni sezioni globali ma solo locali ).
		
		Attenzione: come definisco l'equazione del moto se il fibrato di configurazione non è vettoriale? nel senso: non mi serve tanto la definizione di + e x ma quello che mi serve è l'esistenza di un unico elemento nullo. Se manca questo manca la possibilità di poter definire un'equazione.
		
		Attenzione2: il motion operator credo che debba essere un bundlemorfismo cioè deve essere fiber preserving su ogni fibra determina un endomorfismo. ??? non va bene ho detto che agisce sulle sezioni!

		\begin{notationfix}
		For a given dynamical operator:%Given an equations of motion operator (Correzione CD)
		\begin{displaymath}
			P: \Conf \rightarrow \Conf
		\end{displaymath}
		The space
		\begin{displaymath}
		\gls{Sol} \coloneqq \ker(P) \subset \Conf
		\end{displaymath}
		containing all the smooth solutions is called \emph{"Space of Dynamical Configurations"}.
	\end{notationfix}

	\section{Concrete Realization}
		\subsection{Fields}
			The field systems are a subset of the lagrangian systems:
			\begin{definition}[Linear Fields on curved Background]
				It's a Lagragian system $(E,\Lagrangian)$ such that:
				\begin{itemize}
					\item the configuration bundle $E\xrightarrow{\pi} M$ is a \underline{vector bundle}.
					\item the base manifold $M$ is a \underline{Globally Hyperbolic Spacetime}.
					\item the Euler-Lagrange operator $P= Q_\Lagrangian$ is a \underline{Green Hyperbolic operator}.
					\item For each Cauchy surface $\Sigma \subset M$ can be defined a well-posed Cauchy problem for the motion equation of $P$.\footnote{Green-hyperbolic operators are not necessarily hyperbolic in any PDE-sense and that they cannot be characterized in general by well-posedness of a Cauchy problem. \cite{Terlaky2010} \cite{Bar2010}}
				\end{itemize}
			\end{definition}
		But the other three condition are worth a deeper insight:
		\begin{itemize}
			\item 	\textbf{Vector Bundle Condition}
			\item 	\textbf{Global hyperbolicity condition.}
			\item 	\textbf{Green-Hyperbolicity condition.}
			\item 	\textbf{Cauchy condition.}\\
					While the existence of a Cauchy surface allows to assign the data of initial value problems, the forth condition ensure the well -posedness of the problem for on every Cauchy surface $\Sigma$. I.e:
					\begin{equation}\label{CauchyProblem}
							\begin{cases} P u = 0 \\ u = u_0 \\ \nabla_{\vec{n}}u= u_1 \end{cases}
					\end{equation}
					admit a unique solution $u\in \Gamma(E)$ for all $(u_0, u_1) \in \Gamma (\Sigma )\times \Gamma (\Sigma )$.		
		\end{itemize}

		\begin{observation}
			\begin{center}
				\textbf{Visione Globale}
			\end{center}
			\begin{itemize}
			\item Secondo bar e ginoux per parlare di campo classico non serve specificare nient'altro...
				\begin{itemize}
					\item la condizione di $\exists  1!$ operatore di green di $P$  insieme a quella di Essere un sistema lagrangiano è un requisito minimo  per definire senza ambiguità le parentesi di peierls.
					\item La buona definizione delle parentesi di Peierls è requistio algebrico per portare avanti la quantizzazione algebrica standard (come fa Dappiaggi): \\
					la condizione di green-hyperbolicity ( che garantisce di $\exists 1!\; E^\mp$ ma non che  $\exists 1!$ soluzione del PC) corredata della scelta di un pairing permette di quantizzare secondo lo schema algebrico
					\item La condizione di well-posedness del problema di cauchy da la possibilità di quantizzare secondo lo schema dei dati iniziali
				\end{itemize}
				\item in tutti questi casi la candizione di Globally -hyperbolic per lo spazio tempo sottostante è necessaria
			\end{itemize}
		\end{observation}
		
		\begin{example}
			in adv AQFT ci sono 3 realizzazioni concrete. Klein-Gordon e Proca soddisfano tutte le condizioni precedenti. Anche Dirac ma non è normally Hyperbolic, solo green
		\end{example}

	\subsection{Sistemi a finiti gradi (meccanica geometrica ordinaria)}
		Paragrafo in cui faccio vedere come è possibile vedere un sistema lagrangiano ordinario con un sistema lagrangiano di tipo campo quindi come un sotto-sotto-caso del sistema lagrangiano astratto.

		  % il fibrato non è vettoriale
		  % le configurazioni sono curve
		  % 
		  	Every system with discrete degrees of freedom can be seen as a trivial field system.
			The correspondence is easily done:
			\begin{itemize}
				\item Configuration bundle of the system is the trivial $E= Q \times \Real$ with base manifold $M=\Real$.
				\item The kinematic configuration are $\Conf=C^\infty(\Real,Q)$ i.e.all the possible parametrized functions on $Q$.
				\item The Larangian density is obtained evaluating the ordinary Lagrangian on the lifted curve:
					\begin{equation}
						\Lagrangian  [\gamma] \coloneqq \big( L \circ	\gamma^\textrm{lift} \big) dt  = \Lagrangian(t,\gamma^i,\dot{\gamma}^i)
					\end{equation}
	\end{itemize}
	\begin{Warning}
		devo mettere le conclusione scritte sul primo quaderno insieme a quelle messe nel secondo e poi ripetute a seguito della costruzione di peierls e a quella di quantizzazione ( nei miei appunti io ho fatto ogni singolo passo in generale e poi realizzato per i sistemi campo-curve. Per la stesura finale ho deciso di unire tutto insieme in questo ultimo capitolo (senza ripetere ogni volta che il fibrato è triviale con fibra Q, la varietà base è banalmente globally iperbolic in quanto R. tutti i punti di R sono superfici di cauchy ecc ecc)
	\end{Warning}
	
	\section{Recap Geometric Mechanics}
		
	\begin{Warning}	
	 La visione precedente è molto generale ma ci sono alcune strutture classiche che voglio replicare sul campo come la forma simplettica, le osservabili e le parentesi di poisson.
	 Mi sembra più chiaro vederle dopo aver raccontato queste.
	 
	Quindi devo parlare un po' di meccanica geometrica, di
	\begin{itemize}
		\item Spazio delle Fasi
		\item tautological 1-form
		\item simplectic form
		\item canonical coordinate and darboux theorem
		\item observable as smooth scalar field on the phase space
		\item poisson structure
	\end{itemize}
	\end{Warning}
	
	Attenzione: per quello che mi serve di seguito non mi interessa il discorso della forma simplettica canonica. Quella è la specifica forma simplettica della MECCANICA CLASSICA.\\
	Mi serve solo dire cos'è lo spazio delle fasi e gli osservabili classici.\\
	Faccio accenno al fatto che lo spazio delle fasi classico è naturlmente simplettico.\\
	Definisco quindi l'espressione astratta del sistema hamiltoniano come coppia varietà simplettica + Hamiltoniana.\\
	Fare riferimento che i sistemi hamiltoniani classici posso essere visti come un sottoinsime di quelli lagrangiani (vedi FOMM legendre and hyperregular lagrangia.\\
	Ma, far notare come il processo di quantizzazione della MQO richieda di spostarsi in questo punto di vista più astratto in quanto si basa sulla prescrizione di una diversa espressione delle parentesi di poisson, - quella in grado di implementare le CCR.\\
	
		\begin{proposition}[Possion Structure]
	
		\begin{itemize}
			\item $\{\cdot, \cdot\} = - \{\cdot, \cdot\}$ \emph{Skew-Simmetric}.
			\item $\{\cdot,\{\cdot, \cdot\}\} + \{\cdot,\{\cdot, \cdot\}\} + \{\cdot,\{\cdot, \cdot\}\} = 0$ \emph{Jacobi}
			\item $\{\cdot,\{\cdot, \cdot\}\}= f \{\cdot,\{\cdot, \cdot\}\} + g \{\cdot,\{\cdot, \cdot\}\}$ \emph{Liebniz}
\end{itemize}			
	
	\end{proposition}	
	
	\subsection{Caso Lineare}
		Ricapitolando le novità essenziali che ci saranno utili riguardano il procedimento di quantizzazione sono le seguenti:
		\begin{itemize}
			\item la forma simplettica è definita direttamente sui punti dello spazio delle fasi
			\item la forma simplettica si riproduce sullo spazio degli osservabili lineari ed è compatibile con la poisson
			\item siccome i punti dello spazio delle fasi possono essere messi in corrispondenza biunivoca con le soluzioni (sono i Data) la forma simplettica si trasporta anche su SOL
		\end{itemize}
	
	\section{Peierls Algorithm}
		\begin{Warning}
			\textbf{Possibili migliorie:}
				\begin{itemize}
					\item Aggiungere altre chiacchiere e marketing riguardo le  PB, vedere nelle fonti cosa dicono i sapienti
					\end{itemize}
		\end{Warning}
			\begin{observation}[Peierls Bracket vs Poisson Bracket]
		\begin{Warning}
	Paraphrasing an observation made by Sharan\cite{Sharan2010}:
	
	
		
	While the Poisson bracket between two observables a and b is defined on the whole phase space and is not dependent on the existence of a Hamiltonian, the Peierls bracket refers to a specific trajectory determined by a governing Lagrangian. 
	
	\end{Warning}
	\end{observation}	
	
	\begin{Warning}
		io ho cercato di usare termini diversi come disturbance, perturbation oppure disturbed, perturbed
		ma dare piccole sfumature di differenza
		
		Per la tesi non va bene perchè non sono tutti termini standard! (consiglio di CD)
	\end{Warning}
	
	\subsection{Non Linear Extension}
		non mi è evidente se la rappresentazione in coordinate di un operatore agente sulle sezioni si realizza in modo ovvio, ma non vedo nemmeno ostruzioni! Di sicuro l'operazione è ben definita per gli L.P.D.O visto che la definizione prevede proprio che su ogni carta locale trivilizzante l'operatore sia lineare alle derivate parziali.
	
	\section{Correzioni del Capo}
	>   1) Teorema di Ostro...(non ho capito come si chiama!) per motivare il fatto che ci limitiamo a considerare densità lagrangiane sul primo Jet.

Si tratta di Ostrogradsky, un matematico ucraino ed il problema è noto come instabilità di Ostrogradsky. In realtà non esiste una referenza esplicita in teoria dei campi perché è un problema molto più generale di ODE/PDE. Per assurdo la dimostrazione (e le referenze) date su Wikipedia sono ottime:

	\url{https://en.wikipedia.org/wiki/Ostrogradsky_instability}

     2) posso dire che su qualsiasi fibrato liscio lo spazio delle sezioni liscie è una varietà di Frechet? Se non posso dirlo globalmente potrei ridurmi localmente alla rappresentazione della sezione in carta locale, hai una reference bella dove si prova che lo spazio delle funzioni lisce tra due aperti di $R^n$ è una varietà di frechet?

Dipende tutto dalla fibra tipica del fibrato. Se è un fibrato vettoriale di rango finito, allora sì. Una referenza? Prova il libro di Michor "Convinient calculus" (credo sia il titolo giusto, l'autore è sicuramente lui)

		---oppure... teniamo presente che qualunque sia il fibrato liscio... localmente E = Rq x Rm è banale! quindi le sezioni locali su un aperto sono isomorfe a Cinfty (U) che è frechet...
		secondo me bisogna giocare su questo, non sulla linearità. il problema geodetico non riuscirò mai a vederlo come lineare.. però localmente è dato da tanti operatore uno per ogni aperto dell'atlante della varietà riemmaniana... in questi aperti l'equazione gedoetica prende la forma nota e il PC locale ammette un unica soluzione!
		Poi qua entrerebbe il problema dell'estensione della geodetica (completezza geodetica vedere fomm, jost o abate, \href{http://tinyurl.com/pcxbcqr}{wiki, teorema di hopf-rinow}
		Attenzione: il problema geodetico lo posso definire per aperti... ma la particolarità dell'equazione geodetica è che è compatibile per carte overlapping!
		Se devo specificare la natura dell'operatore del moto generico su fibrato generico ho parecchio da imbarcare... Di sicuro non si passa per attribuire al fibrati di configurazione una sezione "nulla" perchè non ha senso: quale sarebbe la sezione nulla nel caso delle geodetiche? l'annullamento è sempre una questione di carta....

     3)Operatore di Eulero Lagrange è sempre Globalmente Iperbolico nel caso della dinamica di sistemi lagrangiani a gradi di Libertà finiti?

Assolutamente no! Esistono anche lagrangiane non quadratiche nelle velocità ... è proprio il senso del lavoro di Ostrogradsky, ossia ci sono lagrangiane con equazioni del moto altamente instabili.


%\/\/\/\/\/\/\/\/\/\/\/\/\/\/\/\/\/\/\/\/\/\/\/\/\/\/\/\/\/\/\/\/\/\/\/\/\/\/\/\/\/\/\/\/\/\/\/\/\/\/\/\/\/\/\/\/\/\/\/\/\/\/\/\/\/\/\/\/\/\/\/\/
\chapter{Algebraic Quantization}
%\/\/\/\/\/\/\/\/\/\/\/\/\/\/\/\/\/\/\/\/\/\/\/\/\/\/\/\/\/\/\/\/\/\/\/\/\/\/\/\/\/\/\/\/\/\/\/\/\/\/\/\/\/\/\/\/\/\/\/\/\/\/\/\/\/\/\/\/\/\/\/\/
\section{Intro}
	Contemporary quantum field theory is mainly developed as quantization of classical fields. Classical field theory thus is a necessary step towards quantum field theory.\danger (Cito testualmente Mangiarotti, shardanashivly)
	
The \emph{"Quantization process"} has to be considered as an algorithm , in the sense of self-containing succession of instructions, that has to be performed in order to establish a correspondence between a classical field theory and its quantum counterpart.\\
\danger (forse l'nlab esprime la cosa meglio di me \url{http://ncatlab.org/nlab/show/quantization}. Sono d'accordo con il loro approccio ma non voglio usare la loro formulazione perchè in fondo ci sono arrivato anche da solo :P)


\danger\footnote{Frase che non mi piace ma voglio far presente che le realizzazioni dello schema algebrico sono molteplici!}
As said previously, the realization of the Algebraic scheme are many: Fedosov's procedure, by Deformation, Peierls' procedure, by Initial Data etc .
In the next section we review the last two.

\section{Algebraic quantization}
\subsection{Quantization with Peierls}
\danger Questo schema a passi rende bene sul quaderno ma non su latex... riorganizzo tutto in modo che ogni passo venga organizzato in section subsection eccetera.

\danger Attenzione: nello schema nel quaderno, al passo sulla costruzione dello spazio degli osservabili, ho usato una notazione che utilizza il mapping F tra sezioni e funzionali!

	Briefly the procedure can be resumed in few steps:
 	\begin{enumerate}
   		\item Classical Step\\
   			Has to be stated the mathematical structure of the classical theory under examination.
   			\begin{enumerate}
   				\item Kinematics: is encoded in the configuration bundle of the theory.
   					\begin{enumerate}
   						\item Specify the base manifold $M$. \\Has to be a Globally-Hyperbolic Space-time.
   						\item\label{Step:AuxiliaryStructure} Specify the Fiber and the total Space $E$ auxiliary structure, e.g: spin-structure or trasformation laws under diffeomorphism on the base space.\\$E$ has to be at least a vector bundle.
   					\end{enumerate}
   			
   				\item\label{Step:ClassicalDynamicsConditions} Dinamics: has to be specified the local coordinate expression of the motion operator $P: \Gamma^\infty (E)= \Conf \rightarrow \Conf$.
   				   	\begin{enumerate}
   						\item Is $P$ Green-hyperbolic?
   						\item Is $P$ derived from a lagrangian: $P=Q_\Lagrangian$? 
   					\end{enumerate}
   			\end{enumerate}
   			
   		\item Pre-Quantum Step
   		   	\begin{enumerate}
   				\item Pairing: construct a basic bilinear form on the space of kinematic configurations.
   					\begin{enumerate}
   						\item Choose $<\cdot,\cdot>$ a inner product on the bundle $E$.\\ 
						This is a parameter to be guessed. Generally this object is suggested by the auxiliary structures defined on the configuration bundle [\ref{Step:AuxiliaryStructure}].
   						\item Construct the pairing between sections with compact support intersection:
   							\begin{displaymath}
   								(X,Y) = \int_M <X,Y>_x d\mu(x)
   							\end{displaymath}
   						\item is $P$ formally self-adjoint in respect to this pairing?
   					\end{enumerate}
   					
   				\item Classical Observables: construct a set $\Obs$ of suitable classical observables.
   				   	\begin{enumerate}
   						\item Asantz on the off-shell functionals.\\
   							We define a class of linear functional on $\Conf$:
   							\begin{displaymath}
   								\Obs_0 \coloneqq \big\{ F_f: \Conf \rightarrow \Real \:\vert
   								\: F_f(\phi)=\int_M ( f, \phi) d\mu \; , f \in \Gamma_0^\infty(E)	\big\}
   							\end{displaymath}
   							Satisfying the conditions of good definition:
   							\begin{itemize}
   								\item Faithfully representation of the vector space $\Gamma_0^\infty(E)$ (bijective map).
   								\item Separation of the configuration space $\Conf$.   								
   							\end{itemize}
   						\item Domain restriction of the functionals in $\Obs_0$ from $\Conf$ to $\Sol$:
   							\begin{displaymath}
   								\Obs_0\vert_\Sol \coloneqq \big\{F_f\vert_\Sol \big \vert F_f \in \Obs_0 \big\}
   							\end{displaymath}
   							Such that: 
   							\begin{itemize}
   								\item $\Obs_0\vert_\Sol$ met the separability condition.
   								\item The correspondence with  $\Gamma_0^\infty(E)$ is no more injective:
   									\begin{displaymath}
   										\ker \big( \Obs_0 \mapsto \Obs_0\vert_\Sol \big) \coloneqq N \equiv P \big( \Gamma_0^\infty (E) \big)
   									\end{displaymath}
   								\item Define the classical observables space as the quotient:
   									\begin{displaymath}
   										\Obs \coloneqq \frac{\Obs_0\vert_\Sol}{N}
   									\end{displaymath}
   							\end{itemize}   							 
   					\end{enumerate}
   				 
   				 \item Symplectic structure: endow the space $\Obs$ just defined with a bilinear form $\tau$.
   				   	\begin{enumerate}
   						\item Construct the Peierls Brackets between any pair of Lagrangian densities.\\
   							The construction is guaranteed by the requirement in step [\ref{Step:ClassicalDynamicsConditions}].
   						\item Restrict the brackets to the space $\Obs_0$.\\
   							These are very simple Lagrangian functionals:
   							\begin{displaymath}
   								Q_\chi \cdot = (\chi, \cdot) \qquad \forall \chi \in \Obs_0
   							\end{displaymath}
							such that:
							\begin{equation}\label{Def:SymplecticTau}
								\tau ( \phi, \psi) = ( \phi, E \psi)
							\end{equation}
							Conditions of Green-hyperbolicity and formally self-adjoint are sufficient guarantees the good definitions of $\tau$
							\footnote{Has to be noted that the Lagrangian condition is ancillary. This has the purpose to justify the shape of the symplectic form on the classical observables space as consequent from the Peierls bracket.
		\\
	It is frequent\cite{Dewitt1999}\cite{Benini} to overlook to the origin of this object and jump directly to the expression \ref{Def:SymplecticTau}  in term of the Green's operator that no longer present any direct link to the Lagrangian and therefore can be extended to any green-hyperbolic theory.}
						\item Finally the value of $\tau$ is transported to the equivalence classes of $\Obs$ evaluating on the representative:
							\begin{equation}
								\{ [\phi], [\psi]\} \coloneqq \tau(\phi, \psi)
							\end{equation}
							such that:
							\begin{itemize}
								\item $\{\cdot,\cdot\}$ is a simplectic form ( bilinear, antisymmetric, non-degenerate).
								\item Definition is well-posed: do not depend from the representative of the class.
								\item causality axiom
								\item time-slice axiom
							\end{itemize}
   					\end{enumerate}
   			\end{enumerate}
  
   		\item Quantization Step
   		   	\begin{enumerate}
   				\item Quantum Observables Algebra\\
   					A concrete realization is achieved in three step.
   					\begin{enumerate}
   						\item Construct the \emph{Universal Tensor Algebras} of the classical observables.
   						\item
   					\end{enumerate}
   					
   				\item Hadamard State
   				   	\begin{enumerate}
   						\item
   						\item
   					\end{enumerate}
   				 
   			\end{enumerate}
 	\end{enumerate}
 	
	\begin{observation}
		Conditions of Green-hyperbolicity and formally self-adjoint are sufficient guarantees the good definitions of $\tau$.
		Has to be noted that the Lagrangian condition is ancillary. This has the purpose to justify the shape of the symplectic form on the classical observables space as consequent from the Peierls bracket.
		\\
	It is frequent\cite{Dewitt1999}\cite{Benini} to overlook to the origin of this object and jump directly to the expression \ref{Def:SymplecticTau}  in term of the Green's operator that no longer present any direct link to the Lagrangian and therefore can be extended to any green-hyperbolic theory.
	\end{observation}
	
	\begin{Warning}
		Sull'inner product:\\
						Together with the Green-hyperbolicity this condition guarantees that $\exists 1! G^\pm$ and $(G^\pm)^\dagger) = G^\mp$.
				( $\exists 1! E$ causal propagator and $E^\dagger =  -E$.
	\end{Warning}
 	
 	\section{Quantizzazione dati iniziali}
 			\subsubsection{PreQuantum Step.}
	\begin{Warning}
		Si definisce la forma simplettica direttamente su  un opportuno sottoinsieme di data detto "spazio delle fasi classico.\\
		Gli osservabili classici (lineari) vengono definiti di conseguenza.\\
		Si prova che $\Obs$ così costruito è $\simeq$ $\Obs$ con le Peierls.\\
		Ma ciò che non è limpido è il rapporto tra le 2 parentesi costruite sui due spazi isomorfi ( quindi in sostanza su un sistema classico quantizzabile con entrambe le procedure differiscono solo per la diversa prescrizione sulla forma delle brackets)\\
		
		Va Illustrato il parallelismo (p.14 p.16 wald) con i sistemi a finiti gradi. questo esplica come la struttura simplettica definita su TT*Q X TT*Q si confonda con quella delle parentesi di Poisson def su $\Obs \times \Obs = \Sigma^\infty \times \Sigma^\infty$ nel caso di sistemi fisici lineari.
		
		Nell'applicazione di queste 2 procedure  per il campo di Jacobi sono da stressare i seguenti fatti:
		\begin{itemize}
			\item come interviene il fatto che la varietà base sia semplice $M= \Real$ .\\
				I dati iniziali sono numeri
			\item Come interviene l'imput che nello specifico l'eq del moto è data da Jacobi?\\
				(interviene in modo centrale  Riemman-Metrica-Pairing)
		\end{itemize}
		
		Si dice che le Peierls sono automaticamente covarianti (vedi ForgerRomero quando ne parla) mentre quelle sui dati iniziali rompono la covarianza:
		Di base c'è la scelta di una superficie di cauchy su cui dare i dati.
		
		Attenzione ad usare Wald!  è poco formalizzato: e.g.: non parla mai di  green hyperbolic ( anche perchè non serve!) conta solo che il problema di cauchy sia ben posto.
		
		CD parla dei Norm Hyperbolic che sono entrambe le cose!
		
		Quando costruisco C.F.S. scelgo i dati iniziali a supporto compatto.
			\begin{itemize}
				\item Motivazioni a posteriori è chara: garantisce la separabilità nel senso che una misurazione di tutte loro garantisce di esaustire la separabilità (cioè identificare in modo univoco una configurazione)
				\item c'è motivazione a priori? no, è pura praticità: abbastanza semplice da maneggiare + garantisce la separabilità. dunque è scelta minimale: situazione più semplice per fare il pairing. ( se avessi voluto complicarmi la vita avrei preso le distro a supporto compatto
				\item nella scelta non c'entra il fatto che le sup compatto sono funzioni di test, dunque sono il duale delle distribuzioni come dati iniziali.
			\end{itemize}
		\end{Warning}
				\paragraph{Symplectic Structure on the Phase Space}
				[Initial data symplectic form]			
			\begin{Warning}
				Proposta di definizione per dinamiche di ordine più alto..
				\begin{displaymath}
					\Omega \biggr\{ [f_0,\ldots, f_k] , [g_0,\ldots, g_k] \biggr\} =
					\int_\Sigma d\Sigma  \biggr(-\epsilon^{ab}(f_a,g_b) \biggr)
				\end{displaymath}
				gli indici corrono da (0, k-1) dove k è l'ordine... (anche se non ha senso \\
				Non va bene perchè non riesce a tenere conto del campo di Dirac!
			\end{Warning}
	\subsection{Quantum Step}
	   		\paragraph{Hadamard state} 
			Considering the scope of this thesis, we will focus primary on the first step. For a brief account on the construction of the \emph{quantum states} see Ref. \cite{Benini2013}.
			\begin{Warning}
			Non devi certo entrare nei dettagli, ma devi quantomeno accennare al fatto che siano la classe corretta di stati di usare in quanto godono di proprietà rilevanti quali in particolare la possibilità di costruire i polinomi di Wick in modo covariante.
			\end{Warning}

\section{Raccordo tra le due procedure}

	%\subsection{Equivalence of the Brackets}	
	\begin{proposition}[The mapping $\Xi$ is a simplettomorphism.]
		Let be $(\Obs,\tau)$ and $(\Obs_{Lin},\Omega)$ the two classical symplectic spaces according to initial data quantization and Peierls quantization, where
		\begin{displaymath}
			\sigma: \Sol_{sc}\times \Sol_{sc} \rightarrow \Real \qquad 
			\sigma\big( \psi, \phi \big) \int_\Sigma \big( <\nabla_n \psi ,\phi>  - <\psi, \nabla_n \phi> \big) d\Sigma
		\end{displaymath}
		\begin{displaymath}
			\tau: \Obs \times \Obs \rightarrow \Real \qquad
			\tau \big( [f] ,[g] \big) = \int_M < f, Eg> d\mu
		\end{displaymath}
		are the corresponding symplectic forms.\\
		The isomorphism $\Xi: [f] \mapsto E f$ between the two underlying vector spaces preserves the symplectic forms:
		\begin{displaymath}
			\sigma \big( Ef, Eg \big) =  \tau\big( [f], [g] \big)
		\end{displaymath}
		
	
	\end{proposition}
	\begin{proof}
		\danger non vale in generale ma solo per alcuni casi!
	\begin{Warning}
		La formula discende seguendo un ragionamento simile a quello precedente. Nota che vale per P operatore d'onda (o di KG) e non in generale per ogni operatore Green-iperbolico e formalmente autoaggiunto (ad esempio per Dirac non vale, seppur sia un operatore Green iperbolico).
	\end{Warning}
		\begin{align}
		\tau\big( [f], [g] \big) \coloneqq  & \int_M f E h d\textrm{Vol}_M = 
		\int_{\CausalFut(\Sigma)} f \psi d\textrm{Vol}_M	 + \int_{\CausalPast(\Sigma)} f \psi d\textrm{Vol}_M =	\nonumber \\
		=& \int_{\CausalFut(\Sigma)} (P \GreenAdv f) \psi d\textrm{Vol}_M	 + \int_{\CausalPast(\Sigma)} ( P \GreenRet f ) \psi d\textrm{Vol}_M =	 \\
		=&  -\int_\Sigma \big( \nabla_n ( \GreenAdv f) \big) \psi d\Sigma
		+\int_\Sigma \big( ( \GreenAdv f) \big)\nabla_n  \psi d\Sigma +\\
		&+\int_\Sigma \big( \nabla_n ( \GreenRet f) \big) \psi d\Sigma  
		-\int_\Sigma \big( ( \GreenRet f) \big)\nabla_n  \psi d\Sigma =  \\
		=& \int_\Sigma \big( \phi \nabla_n \psi - \psi \nabla_n \phi \big) d\Sigma \coloneqq\sigma (\phi, \psi)
		\end{align}

	\end{proof}

%\/\/\/\/\/\/\/\/\/\/\/\/\/\/\/\/\/\/\/\/\/\/\/\/\/\/\/\/\/\/\/\/\/\/\/\/\/\/\/\/\/\/\/\/\/\/\/\/\/\/\/\/\/\/\/\/\/\/\/\/\/\/\/\/\/\/\/\/\/\/\/\/
\chapter{Jacobi Fields}
%\/\/\/\/\/\/\/\/\/\/\/\/\/\/\/\/\/\/\/\/\/\/\/\/\/\/\/\/\/\/\/\/\/\/\/\/\/\/\/\/\/\/\/\/\/\/\/\/\/\/\/\/\/\/\/\/\/\/\/\/\/\/\/\/\/\/\/\/\/\/\/\/
\section{intro}
	ho definito la geodesica come curva smooth.. in realtà basta essere piecewise C^2!
	
\section{Geodesic Problem as a Mechanical Systems}
	\begin{observation}
		The geodesic system is not simply Lagrangian but also Hamiltonian.
		This property follows from the hyperregularity\cite{Abraham1978} of $L$.
		
		\begin{Warning}
		Anyway we will neglect this fact inasmuch in what follows only the Lagrangian character assumes a role.
		\end{Warning}
	\end{observation}
\section{Geodesic Peierls}
	
	\begin{Warning}
	Introduzione da rividere, dimostro che l'operatore linearizzato senza termine inomogeneo corrisponde all'equazione di Jacobi vera e propria mentre con termine inomogeneo dato dalle E-P del disturbo da l'equazione che definisce la perturbazione ritardata e anticipata.
	\end{Warning}

	\begin{Warning}
		Da dire: espressione in coordinate della lagrangiana, è altmente non lineare perchè implicitamente è $g_{\mu\nu}(\gamma^i(t))$ non polinomiale in $\gamma^i$ ed esplicitamente è quadratica, Mostrare esplicitamente che l'equazione di jacobi per il sistema è effittivamente l'equjazione di jacobi ( questo è triviale se vedi come definisce il campo di jacobi jurgen 
	\end{Warning}	
	
	\begin{Warning}
		Sul quaderno ho scritto un po' di più ma in un modo non adatto alla tesi.! peccato
	\end{Warning}	
		



%\/\/\/\/\/\/\/\/\/\/\/\/\/\/\/\/\/\/\/\/\/\/\/\/\/\/\/\/\/\/\/\/\/\/\/\/\/\/\/\/\/\/\/\/\/\/\/\/\/\/\/\/\/\/\/\/\/\/\/\/\/\/\/\/\/\/\/\/\/\/\/\/
\chapter{Test}
%\/\/\/\/\/\/\/\/\/\/\/\/\/\/\/\/\/\/\/\/\/\/\/\/\/\/\/\/\/\/\/\/\/\/\/\/\/\/\/\/\/\/\/\/\/\/\/\/\/\/\/\/\/\/\/\/\/\/\/\/\/\/\/\/\/\/\/\/\/\/\/\/
	$$L - \mathrm{L} - \mathit{L}  - \mathbf{L} - \mathsf{L} - \mathtt{L}- \mathcal{L}- \mathbb{L}- \mathfrak{L}$$



\end{document}