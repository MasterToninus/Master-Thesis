\usepackage{glossaries}

\makenoidxglossaries

%How to:
%affinchè la voce venga printata nella lista va prima chiamata nel testo come e.g. \gls{Bundle}
%ricordarsi di chiamarlo almeno una volta così, dopo usare il command per evitare il ripetuto hyperref
% anche se si potrebbe evitare visto che il quadratino del link non dovrebbe apparire in stampa

%Advanced Differential Geometry
\newglossaryentry{Bundle}%
{%
	name={\ensuremath{E = (E,\pi , M;Q)}},
	description={ Fiber Bundles $\pi: E\rightarrow M$ with typical fiber $Q$},
    sort={B}
}

\newglossaryentry{Sections}%
{%
	name={\ensuremath{\Gamma^\infty(E)}},
	description={ Smooth sections on the bundle $E$.},
    sort={S}
}


%Geometric Lagrangian Mechanics
	% Kinematic Configurations
	\newglossaryentry{Conf}%
	{%
		name={\ensuremath{\Conf}},
		description={ Kinematic Configurations set}
	}

	%Solutions Space
	\newglossaryentry{Sol}%
	{%
		name={\ensuremath{\Sol}},
		description={ Dynamic Configurations set}
	}

		
	%Lagrangian class
		\newglossaryentry{Lag}%
	{%
		name={\ensuremath{\Lag}},
		description={ Set of Lagrangian densities.}
	}
		
	%Lagrangiana
	\newglossaryentry{Lagrangian}%
	{%
		name={\ensuremath{\Lagrangian}},
		description={ Lagriangian density of the system.}
	}
		
	%Data
	\newglossaryentry{Data}%
	{%
		name={\ensuremath{\Data}},
		description={ Inital Data set.}
	}
		
	%unique solution map
	\newglossaryentry{SolMap}%
	{%
		name={\ensuremath{\SolMap}},
		description={ Map that map a fixed initial data to the unique solution.}
	}
		
	%Classical Observables
	\newglossaryentry{Obs}%
	{%
		name={\ensuremath{\Obs}},
		description={ Set of all classical observables.}
	}

	%Phase Space
	%Classical Observables
	\newglossaryentry{Phase}%
	{%
		name={\ensuremath{\Phase}},
		description={ Phase space.}
	}


