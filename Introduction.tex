\documentclass[Main]{subfiles}
\begin{document}

\chapter*{Introduction \markboth{Introduction}{}}
\addcontentsline{toc}{chapter}{Introduction}


%E' ormai di pubblico dominio il riconoscimento della teoria quantistica dei campi (QFT) di essere la teoria più di successo della storia..
Quantum Field Theory (QFT)  is the synthesis of Quantum Mechanics and Special Relativity and it is the general framework for the description of the physics of relativistic quantum systems.
Its most direct applications, quantum electrodynamics and the standard model of particles, are both been experimentally verified to an outstanding degree of precision and they allowed us to have an almost fully satisfactory and unified description of the electro-weak forces.

In any case it is by no mean a definitive theory. It is clear that the intrinsic quantum description of elementary particles clashes with the the structure of a deterministic theory, as general relativity. 
It is almost unanimously accepted that a quantum theory of gravity is needed in order to reconcile general relativity with the principles of quantum mechanics. Yet, despite countless efforts, a quantum theory of the gravitational interaction remains an open problem.
% Although a quantum theory of gravity is needed in order to reconcile general relativity with the principles of quantum mechanics. 

%l'idea per giungere alla AQFT è per due strade, la prima è quella della rigorizzazione della qed (aximoatic) la seconda è che si tratta del formalismo più generale per estendere la QFT da spazi tempi di minkoski a spazi tempi curvi opportuni.
While quantum field theory has been a well established topic for the past 50 years, the quest of finding a \emph{theory of everything} has often lead the community to neglect the role of two important aspects concerning the QFT.\\
The first is the existence of an intermediate regime, going under the name of \emph{QFT in curved background}, which is expected to provide an accurate description of quantum phenomena in a regions where the effects of curved spacetime may be significant, but effects of quantum gravity itself may be neglected. 
Many successful application of this idea can be found in context of the theory of cosmological inflation or black holes thermodynamic.\\
The second is the construction of a mathematically rigorous description of quantum fields, in particular of their non-perturbative aspects,  based on a sound and shared set of first principles. In other words an \emph{axiomatic foundation of QFT}.

At the moment, \emph{algebraic quantum field theory} (AQFT) is %\ifToninus proven to be the most promising\fi 
a way to complete the picture of the quantum theory of fields regarding the above two aspects.
Its aim is to reach a general and mathematical rigorous description of the foundations of quantum fields on a  sufficiently large class of curved, but fixed, backgrounds.

The algebraic approach, as the majority of contemporary quantum field theory, is developed as a quantization of classical fields.
%Da questo deve riconusciuto che una formalizzazione completa dei fondamenti della QFT non si può esimere da uno costruzione matematicamente precisa dei sistemi campi classici.
%From that should be clear that 
As a starting point, a mathematically rigorous \emph{classical field theory}  is thus a necessary step towards the understanding of the foundations of  the theory.
It has to be noted that classical fields such the force fields of analytical mechanics or the stress tensor in fluid dynamics are not of much interest  insofar QFT is concerned. % più che stress tensor intendevo i CAMPI MATERIALI della fisica dei continui
What is essential to determine is a proper definition of the (pre-quantum) classical analogue of the "fundamental" fields.
%Nell'accezione che riguarda i fondamenti di qft non sono tanto  di interesse i campi classici intesi come campi di forza della meccanica analitica o i tensori definenti i continui materiali, ma bensi potremo dire la versione pre quantistica dei campi fondamentali.
Particularly, it is crucial the identification of the field-theoretic equivalent of the geometric structures underlying the canonical formalism of classical mechanics, namely the \emph{phase space} and an \emph{algebra of classical observables}. 
%In particolar modo è cruciale l'identificazione dell'equivalente field-theoretic delle strutture geometriche proprie del formalismo canonico della meccanica classica, namely the \emph{phase space} and \emph{classical observables algebra}. 
From an abstract point of view  the first is a symplectic manifold, namely a smooth manifold endowed with a non-degenerate 2-form, while the second is a Poisson algebra constituted by functionals on the phase space.

It is important to emphasize that the algebraic quantization is not a unique and well-defined algorithm that reads in a system of classical mechanics and returns a corresponding quantum mechanical system.
Rather it should be seen as a \emph{quantization scheme} which can be realized by several specific procedures.
%La quantizzazione algebrica non si tratta di un unico ben delineato algoritmo that reads in a system of classical mechanics and returns a corresponding system of quantum mechanics, ma bensì di uno schema di quantizzazione che può essere realizzato da differenti procedure specifiche.
All of them are based on a set set of first principles, essentially proposed by Dimock \cite{Dimock1980} as an extension of the axioms of Haag and Kastler formulated on Minkowski spacetime, by prescribing the mathematical structure of the quantum observables algebra.
% che prescrivono la struttura che deve soddisfare l'algebra degli osservabili .
What in which such realizations of the algebraic scheme essentially differ, is in the different identification of the suitable symplectic manifold  to be associated with the pre-quantum version of the field under examination.
%Ciò in cui essenzialmente differiscono queste "realizzazioni" dello schema algebrico,  è nella diversa identificazione delle giusta varietà simplettica da associare alla versione pre quantistica del campo.

The most common way to building these structures requires an explicit choice of \emph{Cauchy surface} in the underlying spacetime. 	
\ifToninus\begin{Warning}
	\begin{itemize}
		\item Frase Originale:\\
		The most common way to building these structures requires an explicit choice of a time function even when no choice is natural. 	
		\item Correzione Cd:\\
		Non è mica vero. La superficie di Cauchy è un oggetto intrinseco e  la sua realizzazione come immersione prima e come elemento di una foliazione poi porta a scegliere una direzione temporale. 
		Tuttavia la quantizzazione esiste a prescindere da quest'ultima scelta.
		\item Conclusione:\\
			Io ho preso questa frase da khavkine (inizio pagina 3), evidentemente non l'avevo capita
	\end{itemize}
\end{Warning}\fi
This leads to a  realization of the algebraic quantization known as \emph{quantization via initial data}.

	
Of much greater interest is the, fully covariant, construction based on the so-called \emph{covariant phase space} and on the \emph{Peierls brackets}.
	The first is defined as the \emph{space of dynamical configurations}, \textit{i.e.}  the infinite-dimensional space of solutions of the equations of motion, while the second is a particular choice of Poisson brackets.% attributable to a field system.
	
	The construction of such Poisson brackets is achieved following what we call the \emph{Peierls' algorithm}, a procedure originally proposed by Rudolf E. Peierls in a seminal paper dated 1952 \cite{Peierls1952}. 
	This is an effective, but rather convoluted, "recipe"  to prescribe a pre-symplectic structure on the space of dynamical configurations. 
	Browsing through the literature, it is clear that the Peierls' construction never received particular attention since its formulation.
This can be ascribed mainly to the lack of a convincing geometric interpretation
which had the effect of limiting its reception often relegating its role to that of a mere  "mathematical trick".

%Cosa vogliamo Fare
The aim of this thesis is to review the original Peierls' procedure in every single aspect adapting it to a more rigorous and modern mathematical formalism.
%Come lo Facciamo
To take a step closer to the comprehension of this object we study the well-known geodesic problem regarding it as a special case of a field-like system.
Essentially this example is noteworthy from two aspects:
\begin{enumerate}
	\item It is a system with discrete degrees of freedom. 
	Its "field configurations" are parametrized curves on a Riemannian manifold and in this sense it represents an example complementary to the basic real scalar field.
	\item This system is dynamically ruled by the well-known \emph{geodesic equation}.
		A typical realization of the algebraic scheme requires to go through the linearization of these, highly non-linear, equations of motion which take the name of \emph{Jacobi equations}.
		The solutions of these linearized equations, named  \emph{Jacobi fields}, are extensively studied from the point of view of differential geometry  ( where they are introduced as a tangent field over a geodesic variation) but they are rarely analysed as a field-like dynamical system.
\end{enumerate}
Since the Jacobi fields can be quantized both according to the \emph{Peierls procedure} and according the \emph{initial data procedure},
we hope that the comparison between the two symplectic spaces thus obtained, allows us to give a new geometric insight on the original Peierls' method.
% ci permetta di fornire un nuovo geometric insight on the original Peierls' method.

\vspace{3mm}
%Sinossi
We briefly summarize the content of the thesis.

The first chapter is devoted to reviewing the mathematical framework  underlying the rigorous formulation of continuous classical systems, starting point of every \ifToninus well-done \fi realizations of the algebraic quantization scheme.
We begin by introducing the notion of \emph{smooth bundles}, these are the suitable objects to encode the kinematical structure of a generic field system.
Subsequently we define the notion of globally hyperbolic spacetime as  the natural arena for the mathematical theory of hyperbolic (systems of) partial differential equations, for which the Cauchy problem is well posed.
Eventually we outline the theory of Green hyperbolic operators, the class of linear differential operators on a vector bundle to which the construction of Peierls brackets, as well as the corresponding quantization procedure, applies.

Chapter 2 is dedicated to introducing the procedure for constructing the Peierls brackets.
In the first part we make use of the mathematical language developed above in order to formalize the correct abstract mechanical system for which the Peierls procedure is well defined. Furthermore we will show how the most familiar mechanical systems, namely the point particle and the free fields over a spacetime, can be treated in a unified way as special cases of the aforementioned abstract system.
\\
In the end we propose an extended version of the original Peierls' algorithm obtained combining the construction proposed in his paper\cite{Peierls1952} with some recent references, mainly \cite{Marolf1993}\cite{Dewitt1999}\cite{Forger2005}\cite{Sharan2010}\cite{Khavkine2014}.
Instead of limiting ourselves only to the case of a scalar field theory, we extend the step-by-step procedure proposed by Peierls to a large class of abstact mechanical systems, not necessarily linear.

In order to pursue the study of the system under investigation it is necessary to introduce the algebraic scheme of quantization.
To this end, the third chapter is focused on presenting two realizations of the algebraic quantization scheme.
The two are distinguished by the different construction of the pre-quantum symplectic space associated to the classical system.
\\
The first one is based on the restriction of the Peierls brackets to the class of \emph{classical observables}, the resulting symplectic form is sometimes prescribed axiomatically \cite{Dewitt1999}\cite{Esposito}\cite{Benini}.\\
The second exploits the hyperbolicity of the equations of motion of the system and it is known as \emph{quantization} via the \emph{initial data}\cite{Wald1994}.

In the last chapter we will apply all the formalism developed so far to the case of the geodesic field.
We construct the Peierls brackets for such system and, as a  concrete example, we carry out all calculations for the specific case of a FRW spacetime with flat spatial sections.
Then we will construct two equivalent pre-quantum symplectic spaces related to the geodesic system following the step-by-step algorithm presented in chapter 3 .
At last we propose a geometric interpretation of the whole Peierls' construction.


\end{document}