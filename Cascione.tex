%\documentclass[a4paper,12pt]{scrartcl}
\documentclass[Main]{subfiles}


\title{Pezzi temporanei e parti eliminate dalla tesi. }
\author{Toninus}
%%Per le Figure
\usepackage[english]{babel}
\usepackage{graphicx}

%simboli matematici strani quali unione disgiunta
\usepackage{amssymb}

%Scrivere Sotto i simboli
\usepackage{amsmath}

%Diagrammi Commutativi
\usepackage{tikz}
\usetikzlibrary{matrix}

%Il simbolo di Identità
\usepackage{dsfont}

%Per riflettere i simboli...
\usepackage{graphicx}


%link iNTERNET
\usepackage{hyperref}

%Enumerate with letters
\usepackage{enumerate}

%Slash over letter
\usepackage{cancel}

%Usare bibiliografia bibtex
%\bibliographystyle{plain}

%Danger sign
\usepackage{fourier}

%:=
\usepackage{mathtools}

%http://tex.stackexchange.com/questions/8625/force-figure-placement-in-text
\usepackage{caption}


%subsection numbering
 \setcounter{tocdepth}{3} % if you want all the levels in your table of contents

%Common symbols
%Common math symbols
	%Number fields
		\newcommand{\Real}{\mathbb{R}}
		\newcommand{\Natural}{\mathbb{N}}
		\newcommand{\Relative}{\mathbb{Z}}
		\newcommand{\Rational}{\mathbb{Q}}
		\newcommand{\Complex}{\mathbb{C}}
	
%equality lingo
	%must be equal
		\newcommand{\mbeq}{\overset{!}{=}} 

% function
	%Domain
		\newcommand{\dom}{\mathrm{dom}}
	%Range
		\newcommand{\ran}{\mathrm{ran}}
	

% Set Theory
	% Power set (insieme delle parti
		\newcommand{\PowerSet}{\mathcal{P}}

%Differential Geometry
	% Atlas
		\newcommand{\Atlas}{\mathcal{A}}
	%support
		\newcommand{\supp}{\textrm{supp}}

	
	
%Category Theory
	%Mor set http://ncatlab.org/nlab/show/morphism
%		\newcommand{\hom}{\textrm{hom}}

%Geometric Lagrangian Mechanics
	% Kinematic Configurations
		\newcommand{\Conf}{\mathtt{C}}
	%Solutions Space
		\newcommand{\Sol}{\mathtt{Sol}}
	%Lagrangian class
		\newcommand{\Lag}{\mathsf{Lag}}
	%Lagrangiana
		\newcommand{\Lagrangian}{\mathcal{L}}
	%Data
		\newcommand{\Data}{\mathsf{Data}}
	%unique solution map
		\newcommand{\SolMap}{\mathbf{s}}
	%Classical Observables
		\newcommand{\Obs}{\mathcal{E}}	
	%Phase Space
		\newcommand{\Phase}{\mathcal{M}}	

		\
		
%Peierls (per non sbagliare più)
		\newcommand{\Pei}{Peierls}

%Accented Letters
\usepackage[utf8]{inputenc}

%Temporaneo, Aggiunta della mia classe teorem... Deve diventare un pacchetto!
\input{../Latex-Theorem/TheoremTemplateToninus.tex}
\begin{document}


%  Titolo
\maketitle
\begin{abstract}
	Sono un accumulatore.\\
	Tutti i mezzi testi, mezze intuizioni che non ho tradotto o a cui non ho trovato posto nella tesi le metto qui dentro.
\end{abstract}


%%%%%%%%%%%%%%%%%%%%%%%%%%%%%%%%%%%%%%%%%%%%%%%%%%%%%%%%%%%%%%%%%%%
\chapter{Prereq Mate}

	\begin{Warning}
	(Primer)\\
	This review is organized as follows: In Section 2 we discuss the key geometrical concepts which lie
at the heart of the construction of quantum field theory on curved backgrounds, particularly the notion of globally hyperbolic spacetimes. In Section 3, we focus instead on the classical description of linear free field theories, whose dynamics is ruled by hyperbolic partial differential equations on vector bundles. In particular we will be interested in those equations which admit advanced and retarded Green operators.
	\end{Warning}



	\section{Fiber Bundles}
	\begin{Warning}
	(Primer)\\
	As a starting point and for the sake of completeness, we recall the definitions of a vector bundle and of
its sections. The latter play a distinguished role since sections represent the natural mathematical object to associate to the physical idea of a classical field. For a detailed discussion of these topics we refer the reader
	\end{Warning}
	\begin{Warning}
	(Advances)
	As a starting point we introduce the building block of any classical and quantum field theory:
	\end{Warning}
	
	
	
		Cosa Serve Dire?
			\subsection{Formal Definition}
			Parto dalla definizione degli smooth.\\
			per la tesi interessano solo i lisci e i vettoriali.\\
			trivialization charts\\
			bundlemorphism\\
			\begin{Warning}
				Forse è necessario parlare del cartesian product. così posso parlare dei tensor bundle e quindi definire il prodotto scalare tra fibrati come una mappa sul primo tensor product?
				vedi def 9 di advances
				
			\end{Warning}			
			
			\subsection{Sections}
			Definizione\\
			
Formal definition ed esempi smooth e bundle (tutto ciò che 

			\subsection{Jet Bundles}
				\url{https://en.wikipedia.org/wiki/Jet_bundle}
				


%%%%%%%%%%%%%%%%%%%%%%%%%%%%%%%%%%%%%%%%%%%%%%%%%%%%%%%%%
	\section{Globally Hyperbolic SpaceTime}
		\begin{Warning}
		(ADVANCEs)\\
		Goal of this section is to introduce all geometric concepts and tools which are necessary to discuss both the classical dynamics and the quantization of a free quantum field on a curved background. We assume that the reader is familiar with the basic notions of differential geometry and, to a minor extent, of general relativity. There-
		\end{Warning}
		
		ricordare di parlare di :
		\\ Sezioni con supporto time, past , future, space - compact (fare schemino come in advances)
		
	\section{Green Hyperbolic Operators}
		\begin{Warning}
		(ADVANCES)\\
		Globally hyperbolic spacetimes play a pivotal role, not only because they do not allowfor pathological situations, such as closed causal curves, but also because they are the natural playground for classical and quantum fields on curved backgrounds. More precisely, the dynamics of most (if not all) models, we are interested in, is either ruled by or closely related to wave-like equations. Also motivated by physics, we want to construct the associated space of solutions by solving an initial value problem. To this end we need to be able to select both an hypersurface on which to assign initial data and to identify an evolution direction. In view of Theorem 1, globally hyperbolic spacetimes appear to be indeed a natural choice. Goal of this section will be to summarize the main definitions and the key properties of the class of partial differential equations, useful to discuss the models that we shall introduce in the next sections. Since this is an overkilled topic, we do not wish to make any claim of being complete and we recommend to an interested reader to consult more specialized books and papers for more details.We suggest for example [33–36], the
		\end{Warning}



%%%%%%%%%%%%%%%%%%%%%%%%%%%%%%%%%%%%%%%%%%%%%%%%%%%%%%%%%%%%%%%%%%%
\chapter{Lagrangian systems e Pierls}
	\section{Presentazione in Astratto}
		Quando parlo della cinematica mi piacerebbe dare indicazioni sulla struttura matematica dello spazio delle configurazioni cinematiche:
			\begin{enumerate}
				\item costituisce una frechet manifold ( gli unici risultati che ho trovato sono quelli di Palais di "non linear global analysis"
				\item le curve parametrizzate sono le variazioni
				\item classi di equivalenza definiscono delle variazioni infinitesime che costituiscono lo spazio tangente allo spazio delle configurazioni cinematiche
				\item questo spazio tangente è isomorfo allo spazio delle sezioni del pullback rispetto alla sezione $\phi\in C$ del verical bundle (vedere forger romero)
				\item il problema dell'atlante e della rappresentazione delle sezioni in carta locale ( da scegliere sia sul total space E che sul base space M)
			\end{enumerate}
		Dovrei fare riferimento al teorema di Ostrowsky per giustificare il fatto che consideriamo solo il primo ordine. le langrangiana con termini cinetici esotici sono instabili ( nel senzo che non ammetto come soluzioni sezioni globali ma solo locali ).

	\section{Concrete Realization}
		\subsection{Fields}
			The field systems are a subset of the lagrangian systems:
			\begin{definition}[Linear Fields on curved Background]
				It's a Lagragian system $(E,\Lagrangian)$ such that:
				\begin{itemize}
					\item the configuration bundle $E\xrightarrow{\pi} M$ is a \underline{vector bundle}.
					\item the base manifold $M$ is a \underline{Globally Hyperbolic Spacetime}.
					\item the Euler-Lagrange operator $P= Q_\Lagrangian$ is a \underline{Green Hyperbolic operator}.
					\item For each Cauchy surface $\Sigma \subset M$ can be defined a well-posed Cauchy problem for the motion equation of $P$.\footnote{Green-hyperbolic operators are not necessarily hyperbolic in any PDE-sense and that they cannot be characterized in general by well-posedness of a Cauchy problem. \cite{Terlaky2010} \cite{Bar2010}}
				\end{itemize}
			\end{definition}
		But the other three condition are worth a deeper insight:
		\begin{itemize}
			\item 	\textbf{Vector Bundle Condition}
			\item 	\textbf{Global hyperbolicity condition.}
			\item 	\textbf{Green-Hyperbolicity condition.}
			\item 	\textbf{Cauchy condition.}\\
					While the existence of a Cauchy surface allows to assign the data of initial value problems, the forth condition ensure the well -posedness of the problem for on every Cauchy surface $\Sigma$. I.e:
					\begin{equation}\label{CauchyProblem}
							\begin{cases} P u = 0 \\ u = u_0 \\ \nabla_{\vec{n}}u= u_1 \end{cases}
					\end{equation}
					admit a unique solution $u\in \Gamma(E)$ for all $(u_0, u_1) \in \Gamma (\Sigma )\times \Gamma (\Sigma )$.		
		\end{itemize}

		\begin{observation}
			\begin{center}
				\textbf{Visione Globale}
			\end{center}
			\begin{itemize}
			\item Secondo bar e ginoux per parlare di campo classico non serve specificare nient'altro...
				\begin{itemize}
					\item la condizione di $\exists  1!$ operatore di green di $P$  insieme a quella di Essere un sistema lagrangiano è un requisito minimo  per definire senza ambiguità le parentesi di peierls.
					\item La buona definizione delle parentesi di Peierls è requistio algebrico per portare avanti la quantizzazione algebrica standard (come fa Dappiaggi): \\
					la condizione di green-hyperbolicity ( che garantisce di $\exists 1!\; E^\mp$ ma non che  $\exists 1!$ soluzione del PC) corredata della scelta di un pairing permette di quantizzare secondo lo schema algebrico
					\item La condizione di well-posedness del problema di cauchy da la possibilità di quantizzare secondo lo schema dei dati iniziali
				\end{itemize}
				\item in tutti questi casi la candizione di Globally -hyperbolic per lo spazio tempo sottostante è necessaria
			\end{itemize}
		\end{observation}
		
		\begin{example}
			in adv AQFT ci sono 3 realizzazioni concrete. Klein-Gordon e Proca soddisfano tutte le condizioni precedenti. Anche Dirac ma non è normally Hyperbolic, solo green
		\end{example}

	\subsection{Sistemi a finiti gradi (meccanica geometrica ordinaria)}
		Paragrafo in cui faccio vedere come è possibile vedere un sistema lagrangiano ordinario con un sistema lagrangiano di tipo campo quindi come un sotto-sotto-caso del sistema lagrangiano astratto.

		  % il fibrato non è vettoriale
		  % le configurazioni sono curve
		  % 
		  	Every system with discrete degrees of freedom can be seen as a trivial field system.
			The correspondence is easily done:
			\begin{itemize}
				\item Configuration bundle of the system is the trivial $E= Q \times \Real$ with base manifold $M=\Real$.
				\item The kinematic configuration are $\Conf=C^\infty(\Real,Q)$ i.e.all the possible parametrized functions on $Q$.
				\item The lagrangian density is obtained evaluating the ordinary Lagrangian on the lifted curve:
					\begin{equation}
						\Lagrangian  [\gamma] \coloneqq \big( L \circ	\gamma^\textrm{lift} \big) dt  = \Lagrangian(t,\gamma^i,\dot{\gamma}^i)
					\end{equation}
	\end{itemize}
	\begin{Warning}
		devo mettere le conclusione scritte sul primo quaderno insieme a quelle messe nel secondo e poi ripetute a seguito della costruzione di peierls e a quella di quantizzazione ( nei miei appunti io ho fatto ogni singolo passo in generale e poi realizzato per i sistemi campo-curve. Per la stesura finale ho deciso di unire tutto insieme in questo ultimo capitolo (senza ripetere ogni volta che il fibrato è triviale con fibra Q, la varietà base è banalmente globally iperbolic in quanto R. tutti i punti di R sono superfici di cauchy ecc ecc)
	\end{Warning}
	
	\section{Recap Geometric Mechanics}
		
	\begin{Warning}	
	 La visione precedente è molto generale ma ci sono alcune strutture classiche che voglio replicare sul campo come la forma simplettica, le osservabili e le parentesi di poisson.
	 Mi sembra più chiaro vederle dopo aver raccontato queste.
	 
	Quindi devo parlare un po' di meccanica geometrica, di
	\begin{itemize}
		\item Spazio delle Fasi
		\item tautological 1-form
		\item simplectic form
		\item canonical coordinate and darboux theorem
		\item observable as smooth scalar field on the phase space
		\item poisson structure
	\end{itemize}
	\end{Warning}
	
	Attenzione: per quello che mi serve di seguito non mi interessa il discorso della forma simplettica canonica. Quella è la specifica forma simplettica della MECCANICA CLASSICA.\\
	Mi serve solo dire cos'è lo spazio delle fasi e gli osservabili classici.\\
	Faccio accenno al fatto che lo spazio delle fasi classico è naturlmente simplettico.\\
	Definisco quindi l'espressione astratta del sistema hamiltoniano come coppia varietà simplettica + Hamiltoniana.\\
	Fare riferimento che i sistemi hamiltoniani classici posso essere visti come un sottoinsime di quelli lagrangiani (vedi FOMM legendre and hyperregular lagrangia.\\
	Ma, far notare come il processo di quantizzazione della MQO richieda di spostarsi in questo punto di vista più astratto in quanto si basa sulla prescrizione di una diversa espressione delle parentesi di poisson, - quella in grado di implementare le CCR.\\
	
		\begin{proposition}[Possion Structure]
	
		\begin{itemize}
			\item $\{\cdot, \cdot\} = - \{\cdot, \cdot\}$ \emph{Skew-Simmetric}.
			\item $\{\cdot,\{\cdot, \cdot\}\} + \{\cdot,\{\cdot, \cdot\}\} + \{\cdot,\{\cdot, \cdot\}\} = 0$ \emph{Jacobi}
			\item $\{\cdot,\{\cdot, \cdot\}\}= f \{\cdot,\{\cdot, \cdot\}\} + g \{\cdot,\{\cdot, \cdot\}\}$ \emph{Liebniz}
\end{itemize}			
	
	\end{proposition}	
	
	\subsection{Caso Lineare}
		Ricapitolando le novità essenziali che ci saranno utili riguardano il procedimento di quantizzazione sono le seguenti:
		\begin{itemize}
			\item la forma simplettica è definita direttamente sui punti dello spazio delle fasi
			\item la forma simplettica si riproduce sullo spazio degli osservabili lineari ed è compatibile con la poisson
			\item siccome i punti dello spazio delle fasi possono essere messi in corrispondenza biunivoca con le soluzioni (sono i Data) la forma simplettica si trasporta anche su SOL
		\end{itemize}
	
	\section{Peierls Algorithm}
		\begin{Warning}
			\textbf{Possibili migliorie:}
				\begin{itemize}
					\item Aggiungere altre chiacchiere e marketing riguardo le  PB, vedere nelle fonti cosa dicono i sapienti
					\end{itemize}
		\end{Warning}
			\begin{observation}[Peierls Bracket vs Poisson Bracket]
		\begin{Warning}
	Paraphrasing an observation made by Sharan\cite{Sharan2010}:
	
	The Poisson bracket determines how one quantity b(t, q, p) changes another quantity a(t, q, p) when it acts as the Hamiltonian or vice-versa. The Peierls bracket, on the other hand, determines how one quantity b(t, q, p) when added to the system Hamiltonian h with an infinitesimal coefficient λ affects changes in another quantity a(t, q, p) and vice-versa, i.e. 	The Peierls bracket is related to the change in an observable when the trajectory on which it is evaluated gets shifted due to an infinitesimal change in the Lagrangian of the system by another Lagragian density.
		
	While the Poisson bracket between two observables a and b is defined on the whole phase space and is not dependent on the existence of a Hamiltonian, the Peierls bracket refers to a specific trajectory determined by a governing Lagrangian. 
	
	\end{Warning}
	\end{observation}	
	
	\subsection{Non Linear Extension}
		non mi è evidente se la rappresentazione in coordinate di un operatore agente sulle sezioni si realizza in modo ovvio, ma non vedo nemmeno ostruzioni! Di sicuro l'operazione è ben definita per gli L.P.D.O visto che la definizione prevede proprio che su ogni carta locale trivilizzante l'operatore sia lineare alle derivate parziali.

	\danger \danger \danger \danger 	\danger \danger \danger \danger 	\danger \danger \danger \danger 
	\section{Dubbi}
	\begin{itemize}
		\item Posso dire che l'operatore di eulero lagrange di un sistema meccanico ordinario è normally iperbolic?
	\end{itemize}



%%%%%%%%%%%%%%%%%%%%%%%%%%%%%%%%%%%%%%%%%%%%%%%%%%%%%%%%%%%%%%%%%%%
\chapter{Test}
	$$L - \mathrm{L} - \mathit{L}  - \mathbf{L} - \mathsf{L} - \mathtt{L}- \mathcal{L}- \mathbb{L}- \mathfrak{L}$$



\end{document}