%^^^^^^^^^^^^^^^^^^^^^^^^^^^^^^^^^^^^^^^^^^^^^^^^^^^^^^^^^^^^^^^^^^^^^^^^^^^^^^^
%		Panoramica del sistema geodetico 1D
%_______________________________________________________________________________

\documentclass{standalone}

\usepackage{amssymb}

\usepackage{tikz}
%\usetikzlibrary{...}
\usepackage{tikz}
\usepackage{pgfplots}
\pgfplotsset{compat=newest}
\usetikzlibrary{calc}

\usepackage[graphics,tightpage,active]{preview}
\PreviewEnvironment{tikzpicture}

%Common symbols
%Common math symbols
	%Number fields
		\newcommand{\Real}{\mathbb{R}}
		\newcommand{\Natural}{\mathbb{N}}
		\newcommand{\Relative}{\mathbb{Z}}
		\newcommand{\Rational}{\mathbb{Q}}
		\newcommand{\Complex}{\mathbb{C}}
	
%equality lingo
	%must be equal
		\newcommand{\mbeq}{\overset{!}{=}} 

% function
	%Domain
		\newcommand{\dom}{\mathrm{dom}}
	%Range
		\newcommand{\ran}{\mathrm{ran}}
	

% Set Theory
	% Power set (insieme delle parti
		\newcommand{\PowerSet}{\mathcal{P}}

%Differential Geometry
	% Atlas
		\newcommand{\Atlas}{\mathcal{A}}
	%support
		\newcommand{\supp}{\textrm{supp}}

	
	
%Category Theory
	%Mor set http://ncatlab.org/nlab/show/morphism
%		\newcommand{\hom}{\textrm{hom}}

%Geometric Lagrangian Mechanics
	% Kinematic Configurations
		\newcommand{\Conf}{\mathtt{C}}
	%Solutions Space
		\newcommand{\Sol}{\mathtt{Sol}}
	%Lagrangian class
		\newcommand{\Lag}{\mathsf{Lag}}
	%Lagrangiana
		\newcommand{\Lagrangian}{\mathcal{L}}
	%Data
		\newcommand{\Data}{\mathsf{Data}}
	%unique solution map
		\newcommand{\SolMap}{\mathbf{s}}
	%Classical Observables
		\newcommand{\Obs}{\mathcal{E}}	
	%Phase Space
		\newcommand{\Phase}{\mathcal{M}}	

		\
		
%Peierls (per non sbagliare più)
		\newcommand{\Pei}{Peierls}

\begin{document}
    \begin{tikzpicture}
       \node (a) at (0,0) %Data
         {
			\begin{tikzpicture}[scale=0.8]
				\begin{axis}[axis lines=none,clip=false]
					%Riquadro che simboleggia Data
					\addplot[dashed,color=gray] coordinates {
						(-1.5,-1.5) (4.5,-1.5) (4.5,4.5)(-1.5,4.5)(-1.5,-1.5)
						};
					\node   [pin={180:$\Data(\Sigma)$},inner sep=0pt]  at (-1.5,4.5){};
				
					%Il piano R2 dei Dati
					\draw[->,color=black] (-1,0) -- (4,0);
					\draw[->,color=black] (0,-1) -- (0,4);
					\node[] at (3.75,3.75) {$\mathbb{R}^2$};
				
					%I dati iniziali
					\draw[-latex,color=green] (0,0) -- (4,-0.7) node [pos=0.5,label={300:{$(a_2,b_2)$}}]  {};    
					\draw[-latex,color=blue] (0,0) -- (3.5,3.2) node [pos=0.5,label={300:{$(a_1,b_1)$}}]  {};

				\end{axis}
			 \end{tikzpicture}
         };
        \node (b) at (8,0)
         {
			\begin{tikzpicture}[remember picture,scale=0.8]
				\begin{axis}[axis lines=none,clip=false]
					% Il riquadro di \Conf
					\addplot[color=black] coordinates {
						(8,6) (0,6) (0,0)(8,0)(8,6)
					}node [pos=1,pin={0:$\Complex^\infty_0 \left(\Real(t)\right)$},inner sep=0pt] {};
					
					% la retta di \Sol
					\addplot[color=red] coordinates {
						(0,2)	
						(8,2)
					}node [pos=1,pin={0:$\Sol$},inner sep=0pt] {};

					% Zero section
					\node(O)[label={270:{$0$}},circle,fill,inner sep=1pt] at (axis cs:1.25,2) {};

					% Fixed Solution
					\node[label={270:{$\vec{X_1}$}},circle,fill,blue,inner sep=1pt] at (axis cs:5,2) {};
					\node[label={270:{$\vec{X_2}$}},circle,fill,green,inner sep=1pt] at (axis cs:3,2) {};


					% Osservabile associato alla soluzione
					\draw[-latex,color=black] (O) -- (4,4) node [pos=1,label={$\chi=E^{-1} \vec{X}$}]  {};  


				\end{axis}
			\end{tikzpicture}
         };
        \node (c) at (4.5,5.5) [ xscale=1.1, yscale=0.9]
         {
         \begin{tikzpicture}
         \begin{axis}[	axis x line=bottom, axis y line=left,
						ticks=none, clip=false,
						xlabel={$t= \gamma(s)$},ylabel={$X(t)$},
						xlabel style={below right},ylabel style={above left},
						xmin=0, xmax=5, ymin=0, ymax=10
						]
 
			% Geodesica di partenza
			\addplot[blue,domain=0:5] ({x},{1.5 + 1.2*x})	
					node[	inner sep=0cm,pos=1,label={0:{$X_1 =a_1+b_1 t$}}]	(G1)	{};
      					
  			% Geodesica Variata
  			\addplot[green,domain=0:5] ({x},{8-0.3*x})
					node[	inner sep=0cm,pos=1,label={0:{$X_2=a_2+b_2 t$}}]	(G2)	{};
	
			%I Dati Iniziali
		
			% Zero point
			\node[label={270:{$\Sigma=0$}},circle,fill,inner sep=1pt] at (axis cs:0,0) {};		
		
		\end{axis}
		\end{tikzpicture}
         };
         
       %Mappe tra gli spazi componenti
        \node (Sol)  at (5.4,-0.75){};
        \node (Data) at (2.7,-0.75){};
              
       \path[->]		
       		(Data.north)    edge[bend left=60,above]  node        {$\SolMap$}  (Sol)
      		(Sol)    edge[bend left=60,below]  node        { \tiny $\left( \left. \cdot \right\vert_\Sigma , \left. D_t( \cdot ) \right\vert_\Sigma  \right)$}  (Data.south);

    \end{tikzpicture}


 

\end{document}