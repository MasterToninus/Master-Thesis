%^^^^^^^^^^^^^^^^^^^^^^^^^^^^^^^^^^^^^^^^^^^^^^^^^^^^^^^^^^^^^^^^^^^^^^^^^^^^^^^
%		Rappresentazione di funzionali lineari di effetto\\
%_______________________________________________________________________________

\documentclass{standalone}

\usepackage{tikz}
%\usetikzlibrary{...}
\usepackage{tikz}
\usepackage{pgfplots}
\pgfplotsset{compat=newest}
\usetikzlibrary{arrows} 

%Funzione 3D
\pgfmathdeclarefunction{F}{2}{\pgfmathparse{ 150 +  #2*10  +#1*2 }}


%Common symbols
%Common math symbols
	%Number fields
		\newcommand{\Real}{\mathbb{R}}
		\newcommand{\Natural}{\mathbb{N}}
		\newcommand{\Relative}{\mathbb{Z}}
		\newcommand{\Rational}{\mathbb{Q}}
		\newcommand{\Complex}{\mathbb{C}}
	
%equality lingo
	%must be equal
		\newcommand{\mbeq}{\overset{!}{=}} 

% function
	%Domain
		\newcommand{\dom}{\mathrm{dom}}
	%Range
		\newcommand{\ran}{\mathrm{ran}}
	

% Set Theory
	% Power set (insieme delle parti
		\newcommand{\PowerSet}{\mathcal{P}}

%Differential Geometry
	% Atlas
		\newcommand{\Atlas}{\mathcal{A}}
	%support
		\newcommand{\supp}{\textrm{supp}}

	
	
%Category Theory
	%Mor set http://ncatlab.org/nlab/show/morphism
%		\newcommand{\hom}{\textrm{hom}}

%Geometric Lagrangian Mechanics
	% Kinematic Configurations
		\newcommand{\Conf}{\mathtt{C}}
	%Solutions Space
		\newcommand{\Sol}{\mathtt{Sol}}
	%Lagrangian class
		\newcommand{\Lag}{\mathsf{Lag}}
	%Lagrangiana
		\newcommand{\Lagrangian}{\mathcal{L}}
	%Data
		\newcommand{\Data}{\mathsf{Data}}
	%unique solution map
		\newcommand{\SolMap}{\mathbf{s}}
	%Classical Observables
		\newcommand{\Obs}{\mathcal{E}}	
	%Phase Space
		\newcommand{\Phase}{\mathcal{M}}	

		\
		
%Peierls (per non sbagliare più)
		\newcommand{\Pei}{Peierls}

\begin{document}
\begin{tikzpicture}
\begin{axis}[axis lines=none,clip=false,view={30}{30},]

	%Funzionale sulle configurazioni
	\addplot3[    surf,    colormap/greenyellow,% mesh, color=gray
	,domain=0:8,y domain=0:6]
	{F(x,y)};

	% Il riquadro di \Conf
	\addplot3[color=black] coordinates {
		(8,6,0) (0,6,0) (0,0,0)(8,0,0)(8,6,0)
		}node [pos=1,pin={270:$\Conf$},inner sep=0pt] {};

	%Valore B(\phi_0)$
	\node[label={135:{\tiny $B(\phi_0)$}},circle,fill,inner sep=1pt] at (axis cs:5,2,{F(5,2)}) {};
	

	% Curve parametrizzate sulla superficie (le spezzo a metà così riesco a mettere bene il vettore tangente
		\addplot3[red,domain=0:2/3,variable=\t,samples y=0] ({3+3*t},{6-6*t},{F(3+3*t,6-6*t)})
			node [pos=0,pin={30:{\tiny $ B(\phi_\epsilon^+)$}},inner sep=0pt] {}
       		node[pos=1,sloped,inner sep=0cm,above,
  	    				anchor=south west,
   	  	 				minimum height=(10+50)*0.02cm,minimum width=(10+50)*0.02cm]
    	  				(P 0) {}		%nodo di phi_0
			;
		 \addplot3[red,domain=2/3:1,variable=\t,samples y=0] ({3+3*t},{6-6*t},{F(3+3*t,6-6*t)});
		% tangent vector
		 \draw[-latex,color=black] (P 0.south west) -- (P 0.south east)node [pos=0.7,label={95:{\tiny $ \EffectOp_\chi^+ B ( \phi_0)$}}] {};


		\addplot3[red,domain=0:(2/3),variable=\t,samples y=0] ({6.33333-2*t},{6-6*t},{F(6.33333-2*t,6-6*t)})
			node [pos=0,pin={30:{\tiny $ B(\phi_\epsilon^-)$}},inner sep=0pt] {}
       		node[pos=1,sloped,inner sep=0cm,above,
  	    				anchor=south west,
   	  	 				minimum height=(10+50)*0.02cm,minimum width=(10+50)*0.02cm]
    	  				(P 1) {}		%nodo di phi_0
			;
		 \addplot3[red,domain=(2/3):1,variable=\t,samples y=0] ({6.33333-2*t},{6-6*t},{F(6.33333-2*t,6-6*t)});
		% tangent vector
		 \draw[-latex,color=black] (P 1.south west) -- (P 1.south east) node [pos=0.75,label={0:{\tiny $ \EffectOp_\chi^- B ( \phi_0)$}}] {}; 


	% L'effetto è la pendenza.. in un certo senso è la componente verticale!
			%Componenti Verticali
			%\draw[-stealth',gray]  (P 0.south west)  --%
      		%	node[right] {$\vec{A_o}$} (P 0.south west |- P 0.south east);

			%\draw[-stealth',yellow]  (P 1.south west)  --%
      		%	node[right] {$\vec{B_o}$} (P 1.south west |- P 1.south east);


	%\\\\\\\\\\\\\\Parte 2d)\\\\\\\\\\\\\\
	% la retta di \Sol
	\addplot[color=red] coordinates {
		(0,2)
		(8,2)
	}node [pos=1,pin={270:$\Sol$},inner sep=0pt] {};
	
	% Zero section
	\node[label={270:{$0$}},circle,fill,inner sep=1pt] at (axis cs:1.25,2) {};
	
	% Fixed Solution
	\node[label={270:{$\phi_0$}},circle,fill,inner sep=1pt] at (axis cs:5,2) {};

	% La curva di \Sol_\epsilon
	\addplot[color=orange,smooth] coordinates {
	(0,4) (2,5.2)(4,4)(6,5)(8,4.5)
	}node [pos=1,pin={270:$\Sol_\epsilon$},inner sep=0pt] {};

	%Variation
	\addplot[color=gray] coordinates {
		(3,6)
		(6,0)
	}node [pos=1,pin={270:$\phi_\epsilon^+$},inner sep=0pt] {};
	\addplot[color=gray] coordinates {
		(6.33333,6)
		(4.33333,0)
	}node [pos=1,pin={270:$\phi_\epsilon^-$},inner sep=0pt] {};
	
	%Perturbation
	\addplot[->] coordinates
           {(5,2) (4,4)}node [pos=0.5,label={180:$\epsilon \eta^+$},inner sep=0pt] {};
	\addplot[->] coordinates
           {(5,2) (6,5)}node [pos=0.5,label={0:$\epsilon \eta^-$},inner sep=0pt] {};


\end{axis}
\end{tikzpicture}
\end{document}