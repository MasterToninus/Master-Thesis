\documentclass[Cascione]{subfiles}
\begin{document}
\chapter{Catchphrases per la presentazione}
	\section{1}
	\begin{itemize}
		\item La spinta a costruire lo spazio simplettico per le teorie di campo classico è stata data da crinkovic witten e zuckerman a partire dagli anni 80
		\item nello spirito del formalismo canonico l'oggetto fondamentale  -necessario e sufficiente - a codificare l'intera struttura matematica di un sistema fisico è la varietà simplettica associato al sistema. Che prende il nome di Spazio delle Fasi.
		\item[$\sim$]Crnkovic aggiunge che dobbiamo essere coordinate -free 
		\item Questo approccio si dimostra vincente nell'ottica delle teorie di quantizzazione . Il motivo è che fornisce la cornice base - grazie alla struttura di poisson che discende automaticamente da esso- su cui costruire l'analogo classico delle strutture quantistiche fondamentali.
		\item Uno dei difetti più fastidiosi del formalismo canonico usuale è la mancanza di covarianza a vista.
		\item la meccanica geometrica sfrutta la geometria moderna per descrivere i sistemi meccanici. E' innegabile che la geometria sia una parte intrinseca della meccanica. ad esempio lo spazio delle configurazioni ammissibili (conformazioni per dirlo alla Tiana) ha la struttura geometrica naturale di varietà liscia.
	\end{itemize}		
	
	
	
	\section{2}
	\begin{itemize}
		\item Per semplificare il discorso consideriamo $\phi$ essere un semplice campo scalare. Nella tesi abbiamo trattato in realtà campi più generali complicando in modo controllato. Non abbiamo trattato le complicazioni tecniche che riguardano le spin structure, libertà di gauge o vincoli.
		\item
	\end{itemize}



	\section{3}
	\begin{itemize}
		\item E' evidente che una costruzione così articolata non si presta ad un'interpretazione geometrica immediata
		\item E' ancora in atto il tentavo di individuare in modo preciso la struttura geometrica delle parti coinvolte.
	\end{itemize}


\chapter{Cose Da dire}
	\subsection{Meccanica Geometrica}
					\begin{itemize}
				\item Raccontare l'approccio canonico coordinate free alla meccanica classica a finiti gradi
					\begin{itemize}
						\item Lo spazio delle fasi
						\item La forma simplettica
						\item lagrangiana hamiltoniana legendre
						\item struttura di Poisson
					\end{itemize}
				\item Dire che un suo aspetto importante è che funge da base per le teorie di quantizzazione
				\item Ci interessa quantizzare i campi quindi ci interessa il formalismo canonico per i campi
				\item Presentare l'approccio al formalismo canonico dei campi in 2 step:
				\item Ok, fin qui abbiamo un insieme. Come lo doto di una struttura simplettica?
			\end{itemize}
			
	\subsection{Formalismo canonico ordinario finiti gradi}
				(usual approach) I passi principali per la costruzione del formalismo canonico per teorie con $N$ gradi di libertà
						\begin{enumerate}
							\item si introducono la coordinate di configuazione $q^j$ e i momenti canonici coniugati $p^j$
							\item si definisce la 2-forma $\omega = dp^i \wedge dq^i$
							\item combinando $p^i,q^j$ in un'unica variablie $Q^I , I = 1,\ldots,2N$, con $Q^i=p^i$ per $i\leq\leq N$ e $Q^i= q^{i-N}$ per $i>N$. E' possibile riguardare $\omega$ come una matrice $2N \times N$  antisimmetrica i cui elementi non nulli sono
								\begin{displaymath}
									\omega_{i , i+N} = -\omega_{i+N,i} = 1
								\end{displaymath}
							\item le parentesi di poisson per 2 funzioni $A(Q^I) , B(Q^J)$ vengono definite come:
								\begin{displaymath}
									\left[ A , B \right] = \omega^{I J} \frac{\partial A}{\partial Q^I} \frac{\partial B}{\partial Q^J}					
								\end{displaymath}
						\end{enumerate}

	\subsection{Recuperare il formalismo canonico non covariante}
			(extra) Recupare il formalismo canonico non covariante anche per i campi costruendo il dato su una superficie di Cauchy (ricordare che al continuo ci sono controesempi che darboux non vale)
			\\
			Lo si fa per sistemi su spazitempi globalmenente iperbolici. in breve sono varietà molto generali in cui è possibile definire per bene i problemi di cauchy
			\\
			\comment{Osservazione chiave di questo punto : spazio delle fasi ordinario è spazio dei dati.. quindi parliamo di Data e Sol}

\subsection{Cappello sul Metodo di Peirls}
{Cappello sul Metodo di Peierls }
			\begin{itemize}
				\item semplificare: usare i grafici già fatti e considerare solo i campi scalari
			\end{itemize}
			His essential insight was to consider the advanced and retarded “effect of one quantity (A) on another (B).” Here, A and B are to
be functions on H. 

			The advanced  and retarded  effects of A on B are then defined by comparing the original system with a new system defined by the action $S_\epsilon = S + \epsilon A$ and the same space H of histories. 
			Under retarded (advanced) boundary conditions for which the solutions $\phi \in S$ and $\phi_\epsilon	\in S_\epsilon$ coincide to the past (future) of the support of A, the quantity $B_0 = B(\phi)$ computed using $\phi$ will in general differ from $B_\epsilon = B(\phi_\epsilon)$ computed using $\phi_\epsilon$.
			For small epsilon, the difference between these quantities defines the retarded (advanced) effect of A on B through:
which depends on the unperturbed solution $\phi$.

\end{document}