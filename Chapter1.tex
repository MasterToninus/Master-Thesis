\documentclass[Main]{subfiles}
\begin{document}

\chapter{Mathematical Preliminaries}
	\begin{Warning}
		Le interazioni matematiche sono complesse  e non triviali (vedi un po' di articoli di introduzione a AQFT per ispirarti)
		
		Tendenzialmente le teorie quantistiche di campi moderne sono di Quantizzazione.. Quindi richiedono di specificare bene la struttura del campo classico (vedi intro di Mangiaratti shardashivly)
		
		Gli strumenti matematici per raccontare la teoria dei campi classici sono essenzialmente 3: Fibrati, S-T G-H, LDOP e GHOP.
		
		IN questo paper non ci soffermeremo sulle strutture del framework puramente quantistico (* algebre e quant'altro).
		
		Diamo per scontato un background di base in Geometria differenziale e derivate esterne (algebre di Grassman? global calculus? non so come chiamarlo!)
		
		Potrei avere la tentazione a provare ad usare un po' di linguaggio basilare delle categorie... la mia fonte è \href{http://katmat.math.uni-bremen.de/acc/acc.pdf}{Joy of Cat}.
	\end{Warning}
	
	\section{Fiber Bundles}
		\subsection{...}
			\begin{Warning}
				Inserire solo i punti salienti del primo capitolo.. Spostare ex primo capitolo spostato nel  repository "dispensarium" come dispensa WIP
			\end{Warning}			
						
		\subsection{Some Topics useful in Physics}
			\subsubsection{Jet Bundles}
			\subsubsection{Tautological one-form and simplectic form.}
	
	\section{Globally Hyperbolic Space-times}
			\begin{Warning}
				Mettere solo le definizioni che uso prese dagli articoli di review delle Fonti
			\end{Warning}	
					\subfile{DefinitionBulk2}		
			
		\section{Green Hyperbolic Operators}
			\begin{Warning}
				Mettere solo le definizioni che uso prese dagli articoli di review delle Fonti
		\end{Warning}	
				\subfile{DefinitionBulk3}
				

\end{document}