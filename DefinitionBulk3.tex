\documentclass[Main]{subfiles}

\begin{document}
	(sono ripetizioni inutili per la tesi, sono informazioni che si ritrovano ovunque... sono informazioni adatta al knowledge base)
Basic Definition in L.P.D.O. on smooth vector sections.
\\
Consider $F=F(M,\pi,V), F'=F'(M,\pi',V')$ two linear vector bundle over $M$ with different typical fiber
	\begin{definition}[Linear Partial Differential operator \footnotesize( of order at most $s\in \Natural_0$)]
		Linear map $L:\Gamma(F)\rightarrow \Gamma(F')$ such that:
		\\
		$\forall p \in M$ exists:
		\begin{itemize}
			\item $(U, \phi)$ local chart on $M$.
			\item $(U, \chi)$ local trivialization of $F$
			\item $(U, \chi')$ local trivialization of $F'$
		\end{itemize}
		for which:
		\begin{displaymath}
			L \big(\sigma \big\vert_U\big) = \sum_{\vert \alpha \vert \leq s} A_\alpha \partial^\alpha \sigma \qquad \forall \sigma \in \Gamma(M)
		\end{displaymath}
	\end{definition}

	\begin{remark}
	(multi-index notation)
	\\
	A multi-index is a natural valued finite dimensional vector $\alpha = ( \alpha_0, \ldots, \alpha_n-1) \in \Natural^n_0$ with $n<\infty$.
	\\
	On $\Real^n$ a general differential operator can be identified by a multi-index:
	\begin{displaymath}
		\partial^\alpha = \prod_{\mu = 0}^{n-1} \partial_\mu ^{\alpha_\mu}
	\end{displaymath}
	(Until the Schwartz theorem holds, the order of derivation is irrelevant.)
	\\
	The order of the multi-index is defined as:
	\begin{displaymath}
		\vert \alpha \vert \coloneqq \sum_{\mu=0}^{n-1} \alpha_\mu
	\end{displaymath}
	\end{remark}

	?????????????????????
	\begin{proposition}[Existence and uniqueness for the Cauchy Problem]
		\begin{hypothesis}
			\begin{itemize}
				\item $\mathbf{M} = (M,g,\mathfrak{o},\mathfrak{t}) $a globally hyperbolic space-time.
				\item $\Sigma \subset M$ a spacelike cauchy surface with future-pointing unit normal vector field $\vec{n}$.
				\item 
			\end{itemize}
		\end{hypothesis}
	\begin{thesis}

	\end{thesis}
	\end{proposition}
	\begin{observation}
	"Green-hyperbolic operators are not necessarily hyperbolic in any PDE-sense and that they cannot be characterized in general by well-posedness of a Cauchy problem.	" \cite{Terlaky2010} \cite{Bar2010}
	\\
	However the existence and uniqueness can be proved for the large class of the \emph{Normally-Hyperbolic Operators}.
	
	\end{observation}

\end{document}