\usetikzlibrary{positioning}
%% helper macros

% The 3D code is based on The drawing is based on Tomas M. Trzeciak's 
% `Stereographic and cylindrical map projections example`: 
% http://www.texample.net/tikz/examples/map-projections/
\newcommand\pgfmathsinandcos[3]{%
  \pgfmathsetmacro#1{sin(#3)}%
  \pgfmathsetmacro#2{cos(#3)}%
}
\newcommand\LongitudePlane[3][current plane]{%
  \pgfmathsinandcos\sinEl\cosEl{#2} % elevation
  \pgfmathsinandcos\sint\cost{#3} % azimuth
  \tikzset{#1/.style={cm={\cost,\sint*\sinEl,0,\cosEl,(0,0)}}}
}
\newcommand\LatitudePlane[3][current plane]{%
  \pgfmathsinandcos\sinEl\cosEl{#2} % elevation
  \pgfmathsinandcos\sint\cost{#3} % latitude
  \pgfmathsetmacro\yshift{\cosEl*\sint}
  \tikzset{#1/.style={cm={\cost,0,0,\cost*\sinEl,(0,\yshift)}}} %
}

\newcommand\DrawLongitudeCircle[2][1]{
  \LongitudePlane{\angEl}{#2}
  \tikzset{current plane/.prefix style={scale=#1}}
   % angle of "visibility"
  \pgfmathsetmacro\angVis{atan(sin(#2)*cos(\angEl)/sin(\angEl))} %
  \draw[current plane,thin,black] (\angVis:1) arc (\angVis:\angVis+180:1);
  \draw[current plane,thin,dashed] (\angVis-180:1) arc (\angVis-180:\angVis:1);
}

\newcommand\DrawLongitudeCircleFancy[2][1]{
  \LongitudePlane{\angEl}{#2}
  \tikzset{current plane/.prefix style={scale=#1}}
   % angle of "visibility"
  \pgfmathsetmacro\angVis{atan(sin(#2)*cos(\angEl)/sin(\angEl))} %
  \draw[current plane,thick,red] (\angVis:1) arc (\angVis:\angVis+180:1);
  \draw[current plane,thick,dashed,red] (\angVis-180:1) arc (\angVis-180:\angVis:1);
}

\newcommand\DrawLongitudeField[2][1]{
  \LongitudePlane{\angEl}{#2}
  \tikzset{current plane/.prefix style={scale=#1}}
   % angle of "visibility"
  \pgfmathsetmacro\angVis{atan(sin(#2)*cos(\angEl)/sin(\angEl))} %
  \draw[current plane,thick,orange] (\angVis:1) arc (\angVis:\angVis+180:1)
		\foreach \p in {0,10,...,100} {
			node[sloped,pos=\p*0.01]
			(N \p){}
    	}
  		;
  \pgfmathsetmacro\angVisB{atan(sin(#2+10)*cos(\angEl)/sin(\angEl))} %
  \draw[current plane,thick,green] (\angVisB:1) arc (\angVisB:\angVisB+180:1)
		\foreach \p in {0,10,...,100} {
			node[sloped,pos=\p*0.01]
			(M \p){}
    	}
  		;


		\foreach \p in {0,10,...,100} {
      		\draw[-latex,color=red] (N \p.center) -- (M \p.center);
 		}	
	 \draw[current plane,thick,dashed,orange] (\angVis-180:1) arc (\angVis-180:\angVis:1);
}


\newcommand\DrawLatitudeCircle[2][1]{
  \LatitudePlane{\angEl}{#2}
  \tikzset{current plane/.prefix style={scale=#1}}
  \pgfmathsetmacro\sinVis{sin(#2)/cos(#2)*sin(\angEl)/cos(\angEl)}
  % angle of "visibility"
  \pgfmathsetmacro\angVis{asin(min(1,max(\sinVis,-1)))}
  \draw[current plane,thin,black] (\angVis:1) arc (\angVis:-\angVis-180:1);
  \draw[current plane,thin,dashed] (180-\angVis:1) arc (180-\angVis:\angVis:1);
}

\newcommand\DrawLatitudeCircleFancy[2][1]{
  \LatitudePlane{\angEl}{#2}
  \tikzset{current plane/.prefix style={scale=#1}}
  \pgfmathsetmacro\sinVis{sin(#2)/cos(#2)*sin(\angEl)/cos(\angEl)}
  % angle of "visibility"
  \pgfmathsetmacro\angVis{asin(min(1,max(\sinVis,-1)))}
  \draw[current plane,thick,green] (\angVis:1) arc (\angVis:-\angVis-180:1);
  \draw[current plane,thick,dashed,green] (180-\angVis:1) arc (180-\angVis:\angVis:1);
}

\tikzset{%
  >=latex,
  inner sep=0pt,%
  outer sep=2pt,%
  mark coordinate/.style={inner sep=0pt,outer sep=0pt,minimum size=3pt,
    fill=black,circle}%
}
\usepackage{amsmath}
\usetikzlibrary{arrows}
\pagestyle{empty}
\usepackage{pgfplots}
\usetikzlibrary{calc,fadings,decorations.pathreplacing}