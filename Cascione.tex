%\documentclass[a4paper,12pt]{scrartcl}
\documentclass[Main]{subfiles}


\title{Pezzi temporanei e parti eliminate dalla tesi. }
\author{Toninus}
%%Per le Figure
\usepackage[english]{babel}
\usepackage{graphicx}

%simboli matematici strani quali unione disgiunta
\usepackage{amssymb}

%Scrivere Sotto i simboli
\usepackage{amsmath}

%Diagrammi Commutativi
\usepackage{tikz}
\usetikzlibrary{matrix}

%Il simbolo di Identità
\usepackage{dsfont}

%Per riflettere i simboli...
\usepackage{graphicx}


%link iNTERNET
\usepackage{hyperref}

%Enumerate with letters
\usepackage{enumerate}

%Slash over letter
\usepackage{cancel}

%Usare bibiliografia bibtex
%\bibliographystyle{plain}

%Danger sign
\usepackage{fourier}

%:=
\usepackage{mathtools}

%http://tex.stackexchange.com/questions/8625/force-figure-placement-in-text
\usepackage{caption}


%subsection numbering
 \setcounter{tocdepth}{3} % if you want all the levels in your table of contents

%Common symbols
%Common math symbols
	%Number fields
		\newcommand{\Real}{\mathbb{R}}
		\newcommand{\Natural}{\mathbb{N}}
		\newcommand{\Relative}{\mathbb{Z}}
		\newcommand{\Rational}{\mathbb{Q}}
		\newcommand{\Complex}{\mathbb{C}}
	
%equality lingo
	%must be equal
		\newcommand{\mbeq}{\overset{!}{=}} 

% function
	%Domain
		\newcommand{\dom}{\mathrm{dom}}
	%Range
		\newcommand{\ran}{\mathrm{ran}}
	

% Set Theory
	% Power set (insieme delle parti
		\newcommand{\PowerSet}{\mathcal{P}}

%Differential Geometry
	% Atlas
		\newcommand{\Atlas}{\mathcal{A}}
	%support
		\newcommand{\supp}{\textrm{supp}}

	
	
%Category Theory
	%Mor set http://ncatlab.org/nlab/show/morphism
%		\newcommand{\hom}{\textrm{hom}}

%Geometric Lagrangian Mechanics
	% Kinematic Configurations
		\newcommand{\Conf}{\mathtt{C}}
	%Solutions Space
		\newcommand{\Sol}{\mathtt{Sol}}
	%Lagrangian class
		\newcommand{\Lag}{\mathsf{Lag}}
	%Lagrangiana
		\newcommand{\Lagrangian}{\mathcal{L}}
	%Data
		\newcommand{\Data}{\mathsf{Data}}
	%unique solution map
		\newcommand{\SolMap}{\mathbf{s}}
	%Classical Observables
		\newcommand{\Obs}{\mathcal{E}}	
	%Phase Space
		\newcommand{\Phase}{\mathcal{M}}	

		\
		
%Peierls (per non sbagliare più)
		\newcommand{\Pei}{Peierls}

%Accented Letters
\usepackage[utf8]{inputenc}

%Temporaneo, Aggiunta della mia classe teorem... Deve diventare un pacchetto!
\input{../Latex-Theorem/TheoremTemplateToninus.tex}
\begin{document}


%  Titolo
\maketitle
\begin{abstract}
	Sono un accumulatore.\\
	Tutti i mezzi testi, mezze intuizioni che non ho tradotto o a cui non ho trovato posto nella tesi le metto qui dentro.
\end{abstract}


%%%%%%%%%%%%%%%%%%%%%%%%%%%%%%%%%%%%%%%%%%%%%%%%%%%%%%%%%%%%%%%%%%%
\chapter{Prereq Mate}
	Quando parlo della cinematica mi piacerebbe dare indicazioni sulla struttura matematica dello spazio delle configurazioni cinematiche:
			\begin{enumerate}
				\item costituisce una frechet manifold ( gli unici risultati che ho trovato sono quelli di Palais di "non linear global analysis"
				\item le curve parametrizzate sono le variazioni
				\item classi di equivalenza definiscono delle variazioni infinitesime che costituiscono lo spazio tangente allo spazio delle configurazioni cinematiche
				\item questo spazio tangente è isomorfo allo spazio delle sezioni del pullback rispetto alla sezione $\phi\in C$ del verical bundle (vedere forger romero)
				\item il problema dell'atlante e della rappresentazione delle sezioni in carta locale ( da scegliere sia sul total space E che sul base space M)
			\end{enumerate}
	
	Dovrei fare riferimento al teorema di Ostrowsky per giustificare il fatto che consideriamo solo il primo ordine. le langrangiana con termini cinetici esotici sono instabili ( nel senzo che non ammetto come soluzioni sezioni globali ma solo locali ).






%%%%%%%%%%%%%%%%%%%%%%%%%%%%%%%%%%%%%%%%%%%%%%%%%%%%%%%%%%%%%%%%%%%
\chapter{Lagrangian systems e Pierls}
	\section{Concrete Realization}
		\subsection{Fields}
			The field systems are a subset of the lagrangian systems:
			\begin{definition}[Linear Fields on curved Background]
				It's a Lagragian system $(E,\Lagrangian)$ such that:
				\begin{itemize}
					\item the configuration bundle $E\xrightarrow{\pi} M$ is a \underline{vector bundle}.
					\item the base manifold $M$ is a \underline{Globally Hyperbolic Spacetime}.
					\item the Euler-Lagrange operator $P= Q_\Lagrangian$ is a \underline{Green Hyperbolic operator}.
					\item For each Cauchy surface $\Sigma \subset M$ can be defined a well-posed Cauchy problem for the motion equation of $P$.\footnote{Green-hyperbolic operators are not necessarily hyperbolic in any PDE-sense and that they cannot be characterized in general by well-posedness of a Cauchy problem. \cite{Terlaky2010} \cite{Bar2010}}
				\end{itemize}
			\end{definition}
		But the other three condition are worth a deeper insight:
		\begin{itemize}
			\item 	\textbf{Vector Bundle Condition}
			\item 	\textbf{Global hyperbolicity condition.}
			\item 	\textbf{Green-Hyperbolicity condition.}
			\item 	\textbf{Cauchy condition.}\\
					While the existence of a Cauchy surface allows to assign the data of initial value problems, the forth condition ensure the well -posedness of the problem for on every Cauchy surface $\Sigma$. I.e:
					\begin{equation}\label{CauchyProblem}
							\begin{cases} P u = 0 \\ u = u_0 \\ \nabla_{\vec{n}}u= u_1 \end{cases}
					\end{equation}
					admit a unique solution $u\in \Gamma(E)$ for all $(u_0, u_1) \in \Gamma (\Sigma )\times \Gamma (\Sigma )$.		
		\end{itemize}

		\begin{observation}
			\begin{center}
				\textbf{Visione Globale}
			\end{center}
			\begin{itemize}
			\item Secondo bar e ginoux per parlare di campo classico non serve specificare nient'altro...
				\begin{itemize}
					\item la condizione di $\exists  1!$ operatore di green di $P$  insieme a quella di Essere un sistema lagrangiano è un requisito minimo  per definire senza ambiguità le parentesi di peierls.
					\item La buona definizione delle parentesi di Peierls è requistio algebrico per portare avanti la quantizzazione algebrica standard (come fa Dappiaggi): \\
					la condizione di green-hyperbolicity ( che garantisce di $\exists 1!\; E^\mp$ ma non che  $\exists 1!$ soluzione del PC) corredata della scelta di un pairing permette di quantizzare secondo lo schema algebrico
					\item La condizione di well-posedness del problema di cauchy da la possibilità di quantizzare secondo lo schema dei dati iniziali
				\end{itemize}
				\item in tutti questi casi la candizione di Globally -hyperbolic per lo spazio tempo sottostante è necessaria
			\end{itemize}
		\end{observation}
		
		\begin{example}
			in adv AQFT ci sono 3 realizzazioni concrete. Klein-Gordon e Proca soddisfano tutte le condizioni precedenti. Anche Dirac ma non è normally Hyperbolic, solo green
		\end{example}

	\section{Sistemi a finiti gradi (meccanica geometrica ordinaria)}
		Paragrafo in cui faccio vedere come è possibile vedere un sistema lagrangiano ordinario con un sistema lagrangiano di tipo campo quindi come un sotto-sotto-caso del sistema lagrangiano astratto.

		  % il fibrato non è vettoriale
		  % le configurazioni sono curve
		  % 
		  	Every system with discrete degrees of freedom can be seen as a trivial field system.
			The correspondence is easily done:
			\begin{itemize}
				\item Configuration bundle of the system is the trivial $E= Q \times \Real$ with base manifold $M=\Real$.
				\item The kinematic configuration are $\Conf=C^\infty(\Real,Q)$ i.e.all the possible parametrized functions on $Q$.
				\item The lagrangian density is obtained evaluating the ordinary Lagrangian on the lifted curve:
					\begin{equation}
						\Lagrangian  [\gamma] \coloneqq \big( L \circ	\gamma^\textrm{lift} \big) dt  = \Lagrangian(t,\gamma^i,\dot{\gamma}^i)
					\end{equation}
	\end{itemize}


	\danger \danger \danger \danger 	\danger \danger \danger \danger 	\danger \danger \danger \danger 
	\section{Dubbi}
	\begin{itemize}
		\item Posso dire che l'operatore di eulero lagrange di un sistema meccanico ordinario è normally iperbolic?
	\end{itemize}



%%%%%%%%%%%%%%%%%%%%%%%%%%%%%%%%%%%%%%%%%%%%%%%%%%%%%%%%%%%%%%%%%%%
\chapter{Test}
	$$L - \mathrm{L} - \mathit{L}  - \mathbf{L} - \mathsf{L} - \mathtt{L}- \mathcal{L}- \mathbb{L}- \mathfrak{L}$$



\end{document}