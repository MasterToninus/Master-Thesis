\documentclass[Main]{subfiles}
\begin{document}
\chapter{Catchphrases per la presentazione}

	\section{1}
	\begin{itemize}
		\item La spinta a costruire lo spazio simplettico per le teorie di campo classico è stata data da crinkovic witten e zuckerman a partire dagli anni 80
		\item nello spirito del formalismo canonico l'oggetto fondamentale  -necessario e sufficiente - a codificare l'intera struttura matematica di un sistema fisico è la varietà simplettica associato al sistema. Che prende il nome di Spazio delle Fasi.
		\item[$\sim$]Crnkovic aggiunge che dobbiamo essere coordinate -free 
		\item Questo approccio si dimostra vincente nell'ottica delle teorie di quantizzazione . Il motivo è che fornisce la cornice base - grazie alla struttura di poisson che discende automaticamente da esso- su cui costruire l'analogo classico delle strutture quantistiche fondamentali.
		\item Uno dei difetti più fastidiosi del formalismo canonico usuale è la mancanza di covarianza a vista.
		\item la meccanica geometrica sfrutta la geometria moderna per descrivere i sistemi meccanici. E' innegabile che la geometria sia una parte intrinseca della meccanica. ad esempio lo spazio delle configurazioni ammissibili (conformazioni per dirlo alla Tiana) ha la struttura geometrica naturale di varietà liscia.
			 
		
	\end{itemize}		
	
	
	
	\section{2}
	\begin{itemize}
		\item Per semplificare il discorso consideriamo $\phi$ essere un semplice campo scalare. Nella tesi abbiamo trattato in realtà campi più generali complicando in modo controllato. Non abbiamo trattato le complicazioni tecniche che riguardano le spin structure, libertà di gauge o vincoli.
		\item
	\end{itemize}



	\section{3}
	\begin{itemize}
		\item E' evidente che una costruzione così articolata non si presta ad un'interpretazione geometrica immediata
		\item E' ancora in atto il tentavo di individuare in modo preciso la struttura geometrica delle parti coinvolte.
	\end{itemize}






\end{document}