\documentclass[Main]{subfiles}
\begin{document}

\chapter*{Conclusions}
\addcontentsline{toc}{chapter}{Conclusions}

%DIre per l'ennesima volta cosa si è fatto..
Il confronto con le numerose fonti analizzate ci ha permesso di sintetizzare, in un linguaggio più moderno e rigoroso,  l'algoritmo  di costruzione  delle parantesi proposto da Peierls per una classe di sistemi abbastanza generale: Lagrangiani con operatore del moto Green-Iperbolico e con Operatori di Green unici.

Abbiamo sottolineato come da questa costruzione si possano ottenere in modo pressochè immediato la struttura simplettica di pre-quantizzazione che usualmente viene assunta come un postulato negli schemi di quantizzazione algebrica.

Abbiamo realizzato le parentesi di Peierls e due schemi di quantizzazione algebrica per l'esempio insolito del \emph{Moto Geodetico} e fatto emergere un legame tra le forme simplettiche ottenute tramite queste due differenti costruzioni.

To chapter 


\vspace{2mm}%Possibili estensioni del Lavoro


	Great emphasis should be pointed on the reproposition of the construction procedure  originally proposed by Peierls in a more rigorous and modern fashion, stating the equivalence with the more recent presentation suggested by DeWitt, and connecting to the modern abstract approach\cite{Khavkine} involving variational bicomplex.
	Currently there's lack of bibliography on the theme  that hinders the recognition of the role of this object among the schemes of Algebraic quantization.


%second

 Extend the work of the paper of  Forger and Romero\cite{Forger} from the point of view of differential calculus on infinite dimensional manifolds.
	The main advantage of this approach, would be to make clear the parallelism between the geometrical mechanics of ordinary finite dimensional systems and fields. To this extent the calculus of variation will assume a pivotal role since the variations could be seen as tangent vectors over the manifold of  kinematic configurations.
	Essentially, I propose to review the key findings in non-linear global analysis from the point of view of the geometric mechanics of system with infinite degrees of freedom.
	A good starting point might be the work of Palais\cite{Palais} on the analysis of smooth sections spaces.
	
	\end{document}